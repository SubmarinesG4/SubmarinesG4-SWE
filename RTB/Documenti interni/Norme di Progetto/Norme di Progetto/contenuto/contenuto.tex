%Contenuto del documento
%Introduzione
\section{Introduzione}
\subsection{Scopo del Documento}
Lo scopo di questo documento è di definire le norme, le convenzioni e le procedure adottate da tutti i membri di \groupName{}, in modo da poter definire un metodo di lavoro comune. Tutti i membri sono tenuti a visionare periodicamente tale documento e a rispettare tutte le norme in esso presenti. Per la stesura viene adottata una filosofia incrementale, quindi il documento allo stato attuale è incompleto, con l'aspettativa di avere un processo normato prima del suo avvio, considerando che, in generale, ogni norma può essere soggetta a cambiamenti.

\subsection{Scopo del Prodotto}
L'obiettivo dell'azienda Zero12\glo{} è la creazione di un sistema software costituito da una Webapp. Lo scopo del prodotto è quello di fornire un sistema di localizzazione dei testi per app e webapp. Molti applicativi moderni devono essere multi-lingua in modo da essere disponibili in un mercato internazionale, diventa quindi necessario capire come gestire le traduzioni ed è esattamente questo l’obiettivo che l’applicativo si pone di raggiungere.

\subsection{Glossario}
\gloDesc{}

\subsection{Riferimenti}
\subsubsection{Riferimenti normativi}
\begin{itemize}
\item Presentazione del capitolato\glo{} 4 - Zero12 Progettazione e sviluppo di una Piattaforma di Localizzazione Testi 

\href{https://www.math.unipd.it/~tullio/IS-1/2022/Progetto/C4.pdf}{https://www.math.unipd.it/~tullio/IS-1/2022/Progetto/C4.pdf}
\end{itemize}

\subsubsection{Riferimenti informativi}
\begin{itemize}
\item Slide T03 - Corso di Ingegneria del Software, UniPD a.a. 2022-2023 - Il ciclo di vita del SW

\href{https://www.math.unipd.it/~tullio/IS-1/2022/Dispense/T03.pdf}{https://www.math.unipd.it/~tullio/IS-1/2022/Dispense/T03.pdf}
\item Slide T04 - Corso di Ingegneria del Software, UniPD a.a 2022-2023 - Gestione di progetto

\href{https://www.math.unipd.it/~tullio/IS-1/2022/Dispense/T04.pdf}{https://www.math.unipd.it/~tullio/IS-1/2022/Dispense/T04.pdf}
\item Slide T10 - Corso di Ingegneria del Software, UniPD a.a 2022-2023 - Verifica e validazione: introduzione

\href{https://www.math.unipd.it/~tullio/IS-1/2022/Dispense/T10.pdf}{https://www.math.unipd.it/~tullio/IS-1/2022/Dispense/T10.pdf}
\end{itemize}

\pagebreak

\section{Processi Primari}
\subsection{Fornitura}

\subsubsection{Scopo}
In questa sezione vengono illustrati i documenti e i metodi adoperati per realizzare il processo\glo{} di fornitura, definito secondo lo standard ISO/IEC/IEEE 12207:1995.

\subsubsection{Descrizione}
Il processo di fornitura determina ogni compito e attività necessaria allo svolgimento del progetto. Tale processo permette di risolvere e sanare ogni dubbio legato a ciò che il proponente\glo{} si aspetta di vedere; e verrà avviato solamente in seguito alla comprensione delle richieste del proponente. Seguiranno uno studio di fattibilità e la sigla del contratto con l'azienda interessata, la quale impegnerà il gruppo nella progettazione e realizzazione del prodotto software finale.

\subsubsection{Rapporti con il proponente}
\paragraph{Azienda}
L’azienda Zero12 accompagna le aziende clienti in un percorso di innovazione che si pone i
seguenti obiettivi: digitalizzare i processi, creare soluzioni software di engagement, semplici e
indispensabili, per i clienti (interni o esterni). Tutto quello che l’azienda svolge è “Cloud Native\glo{}” e basato su tecnologia “Amazon Web Services\glo{} (AWS)”.

\paragraph{Materiale fornito}
Zero12 ha fornito, oltre al capitolato, una lista delle tecnologie da utilizzare e consigliate per lo
sviluppo del prodotto finale e una serie di corsi di formazione sulle tecnologie AWS, le cui date
saranno definite dall’azienda stessa. In particolare, i corsi di formazione si pongono l'obiettivo di fornire una base solida su: AWS e infrastruttura, front-end\glo{}, back-end{}.

\paragraph{Supporto}
L’azienda ha fornito una lista di tecnologie consigliate per lo sviluppo del progetto e nel caso si
rivelassero quelle scelte dal gruppo, Zero12 si mette a disposizione per un supporto in caso di
difficoltà. Essendo tali tecnologie quelle utilizzate dall’azienda il supporto si rivela molto più
semplice rispetto a tecnologie da loro non utilizzate.

\paragraph{Comunicazione}
Per la comunicazione con il proponente, Zero12 ha aperto un canale Slack\glo{} al quale sono stati
invitati tutti i membri di \groupName{}. Tale canale viene utilizzato per le comunicazioni dirette
con il Project Manager di Zero12, Michele Massaro, che si mette a disposizione per eventuali
dubbi emersi o questioni che richiedono un suo riscontro.

\subsubsection{Documenti}
In questa sezione sono riportati i documenti prodotti in questa fase e i relativi strumenti a supporto della stesura.

\paragraph{Piano di Progetto}
Il piano di progetto è un documento che contiene la pianificazione dei tempi, l'analisi dei rischi, il consuntivo di periodo, la data di consegna e i costi previsti.

\paragraph{Piano di Qualifica}
Il piano di qualifica deve contiene tutte le modalità adottate dal \roleVerifier{} durante la verifica e validazione assicurando che la qualità dei processi e delle risorse rispetti le aspettative. 

\paragraph{Strumenti}
Per gli strumenti adottati per realizzare i processi di questa fase si fa riferimento alle sezioni 3.1.2 (Strumenti per la stesura) e 4.1.4 (Strumenti collaborativi).  

\subsection{Sviluppo}
\subsubsection{Scopo}
In questa sezione vengono presentati i compiti e le attività da svolgere relative allo sviluppo del prodotto software, tra cui le norme e le convenzioni adottate per la composizione di questo processo.

\subsubsection{Descrizione}
Di seguito sono elencate le attività che compongo il processo di sviluppo:
\begin{itemize}
\item Analisi dei requisiti;
\item Progettazione;
\item Codifica.
\end{itemize}

\subsubsection{Analisi dei Requisiti}
\paragraph{Scopo}
Il documento \docNameAdR definisce nel dettaglio l'analisi dei requisiti e i possibili scenari e casi d'uso che tratta. L'obiettivo è quello di analizzare accuratamente quanto presente nel capitolato al fine di individuare tutti i requisiti e i casi d'uso che il proponente richiede per la realizzazione del prodotto.

\paragraph{Requisiti}
I requisiti vengono raccolti attraverso le seguenti fasi:
\begin{itemize}
\item Lettura attenta del capitolato;
\item Confronto interno con il gruppo;
\item Confronto con il proponente;
\end{itemize}
Tali fasi permettono di analizzare nel modo più corretto possibile i requisiti ottenendo infine una conferma fondamentale e necessaria da parte del proponente.

\paragraph{Classificazione dei requisiti}


\paragraph{Casi d'uso}


\paragraph{Classificazione dei casi d'uso}


\paragraph{UML}


\subsubsection{Progettazione}
Lo scopo della progettazione è quello di determinare le caratteristiche che il prodotto finale deve avere, incorporando le parti ottenute durante l'analisi dei requisiti.

\paragraph{Obiettivo}
Il fine della progettazione è la realizzazione dell'architettura di sistema inizialmente realizzata da: un "Proof of Concept\glo{}"  sviluppato come una versione prototipale del sistema; la  "Technology Baseline"; e tramite il documento tecnico  "Product Baseline".

\paragraph{Technology Baseline}
In accordo con il proponente vengono motivate le tecnologie, i framework\glo{} e le librerie selezionate per la realizzazione del prodotto. Dimostra la fattibilità degli obiettivi prestabiliti.
Il \docNamePoC{} raggruppa tutte le tecnologie adottate per la realizzazione del prodotto finale e ne dimostra la compatibilità e adeguatezza alle richieste del proponente.

\paragraph{Product Baseline}
Questa fase illustra le linee guida architetturali del prodotto. Oltre all’evoluzione dei documenti sopracitati, in questa fase risulterà necessario anceh il "Manuale utente".

\subsubsection{Codifica}
Sezione che verrà sviluppata durante la fase di codifica dell'applicativo software..

\section{Processi di Supporto}
\subsection{Documentazione}
\subsubsection{Verbali}
Si è deciso di creare un template per i verbali interni ed esterni in modo da facilitare notevolmente la loro compilazione. Tale template viene messo a disposizione nello spazio di archiviazione  Google Drive\glo{} collegato alla mail del gruppo, in modo che possa essere reperito da chi incaricato per la stesura di un nuovo verbale.
I verbali hanno il compito di riassumere quanto discusso nel relativo meeting in modo che possa essere recuperato anche da chi non era presente a quell'incontro.
Chi incaricato della stesura del verbale dovrà riportare i punti salienti del meeting accompagnati da una breve descrizione.
I verbali, interni ed esterni, sono gli unici documenti che non verranno sottoposti alle regole di versionamento, revisione ed approvazione (rispettivamente illustrate nelle sezioni 3.3, 3.4 e
3.6 di questo documento). I verbali verranno stilati durante gli incontri interni ed esterni e saranno soggetti di una verifica e di una approvazione istantanea subito dopo tali incontri.

\paragraph{Verbali interni}
I verbali interni sono dedicati alle discussioni dei \groupName{}, effettuate al termine di ogni sprint. Prima di ogni meeting verrà scelta la persona responsabile della redazione del verbale. 

\paragraph{Verbali esterni}
I verbali esterni sono dedicati agli incontri con il proponente.
Tali incontri potranno essere richiesti dai \groupName{}, qualora fosse ritenuto necessario esporre a Zero12 dei quesiti riguardo il progetto, oppure potranno essere richiesti dal proponente stesso.

\subsubsection{Strumenti per la stesura}
Per quanto riguarda la stesura della documentazione si sono adottati diversi strumenti in base allo specifico documento e alle necessità che esso presenta. 
Si riportano qui di seguito i diversi software utilizzati in
questo ambito: Google Docs\glo{}, riservato alla stesura dei verbali interni ed esterni; e LaTeX\glo{} per la
stesura della documentazione principale, tra cui: 
\begin{itemize}
\item \docNameNdP{};
\item \docNamePdP{};
\item \docNameAdR{};
\item \docNamePdQ{};
\item \docNameGlo{}.
\end{itemize}
Per entrambi questi strumenti sono stati definiti dei
template per la documentazione che dovranno essere considerati la base comune per la creazione di un nuovo documento dei \groupName{}.

\subsubsection{Versionamento}
La versione permette di capire in che stato si trova il documento. Il numero di versione presenta il seguente formato: \textbf{X.Y.Z}, dove:
\begin{itemize}
    \item \textbf{X} rappresenta una versione approvata dal 
    responsabile di 	progetto; il documento risulta pronto per il 
    suo rilascio non appena raggiunge la versione 1.0.0;
    \item \textbf{Y} rappresenta una versione del documento 
    verificata da uno dei verificatori in carica; il numero 
    incrementa ad ogni verifica del documento e tale cifra si
    azzera ad ogni aumento di X;
    \item \textbf{Z} rappresenta una versione in fase di modifica
    da parte dei redattori del documento, essa aumenta ad ogni 
    modifica e si azzera all’aumentare di Y.
\end{itemize}

\subsubsection{Revisione}
Ogni sezione del corpo del documento è rivista da un verificatore del gruppo che non sia il
redattore della parte in verifica. Una volta verificata la sezione in esame sarà possibile
incrementare il numero Y, il quale indicherà poi lo stato della verifica.

\subsubsection{Approvazione}
Per poter essere approvato, uno specifico documento deve raggiungere quantomeno la
versione 1.0.0 che indica che il documento è stato approvato e può essere rilasciato per la
prima volta.

\section{Processi Organizzativi}
\subsection{Gestione Organizzativa}

\subsubsection{Scopo}

\subsubsection{Descrizione}

\subsubsection{Ruoli di Progetto}
\paragraph{Responsabile di progetto}

\paragraph{Amministratore}

\paragraph{Analista}

\paragraph{Progettista}

\paragraph{Programmatore}

\paragraph{Verificatore}


\subsubsection{Strumenti collaborativi}
\paragraph{GitHub}
Si è deciso di utilizzare il gestore di repository\glo{} messa a disposizione da GitHub\glo{} come base di
archiviazione di tutti gli elementi che compongono il progetto didattico. È stata creata un’organizzazione visitabile al seguente link: \href{https://github.com/SubmarinesG4}{https://github.com/SubmarinesG4}.
Ciascun membro del team presenta l’etichetta di “owner” e pertanto dispone di tutte le autorizzazioni legate a tale etichetta; in particolare quella di popolare la repository.
La repository si presenta con una struttura a n branch, tutti ad uso interno, volti ad effettuare caricamenti di versioni di documentazione/codice ancora non approvati. Il branch principale è denominato "main" e contiene tutta la documentazione/codice approvato e verificato. 
La repository è organizzata in diverse cartelle, ciascuna delle quali è dedicata ad un elemento o sottogruppo specifico (es. cartella "Documentazione interna") che, tramite il nome, indica cosa vi si trova all'interno. 

\paragraph{YouTrack}
Per la gestione dei task del progetto software si è deciso di utilizzare YouTrack\glo{}.
Questo strumento permette di gestire in modo centralizzato tutte le attività connesse alla
gestione di un progetto software. L'elemento chiave è la Issue\glo{} a cui è associata un task da eseguire.
All'interno di Youtrack sono presenti i Projects che rappresentano macro task da svolgere (ad esempio : un project viene creato per ogni tipologia di documentazione richiesta).
Per avere una visione d'insieme dell'avanzamento del progetto software sono presenti le "Agile boards\glo{}" e "Gantt chart\glo{}". I quali insieme all'integrata funzione di tracciamento del tempo, a colpo d'occhio, si è in grado di verificare l'avanzamento del progetto.
Tutte queste funzioni sono automatizzate e gestite dalla piattaforma. Questo permette ai componenti del gruppo di concentrarsi nello sviluppo invece che nella gestione del progetto.

\subsection{Gestione degli incontri}
\subsubsection{Piattaforma}
Come luoghi di incontro vengono utilizzate due piattaforme in base alla tipologia di incontri da effettuare; Discord\glo{} per gli incontri interni e, generalmente, Google Meet\glo{} per gli incontri con il proponente.

\subsubsection{Date}
Tenendo conto degli impegni individuali di ciascun membro dei \groupName{} si è deciso di effettuare quantomeno un incontro interno settimanale, con data e ora da decidersi, per discutere dei progressi fatti e per fissare ciò che si dovrà fare di settimana in settimana. Tale incontro verrà fissato tramite il canale Telegram\glo{} del gruppo; se per qualche motivo alcuni membri non riuscissero a parteciparvi, saranno informati attraverso Telegram e per entrare nel dettaglio potranno recuperare il relativo verbale.

\subsubsection{Argomenti}
Ogni incontro viene suddiviso in tre parti principali: la prima in cui si aggiorna il gruppo su quanto svolto individualmente fino a quel momento; la seconda in cui si decide cosa fare nel prossimo periodo, assegnando delle priorità alle diverse "cose da fare"; e la terza dedicata a domande o dubbi su come procedere.