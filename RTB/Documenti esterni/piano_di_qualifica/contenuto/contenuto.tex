%Contenuto del documento
%Introduzione
\section{Introduzione}
    \subsection{Scopo del documento}
    \subsection{Scopo del prodotto}
    \subsection{Glossario}
    Per chiarezza c'è un documento "Glossario v.1.0.0" presente nella documentazione che va a chiarire tutti i termini e le espressioni che possono risultare ambigue. Questi termini avranno il pedice 'G', esempio "parola\textsubscript G".
In caso venga citato un documento verrà inserito il pedice 'D', esempio "documento\textsubscript D".
    \subsection{Standard di progetto}
    \subsection{Riferimenti}
        \subsubsection{Riferimenti Normativi}
            \begin{itemize}
            \item \emph{NormeDiProgetto-v1.0.0};
            \end{itemize}
        \subsubsection{Riferimenti Formativi}
            \begin{itemize}
            \item \emph{Processi di ciclo di vita - Materiale didattico del corso di Ingegneria del Software: https://www.math.unipd.it/~tullio/IS-1/2021/Dispense/T03.pdf};
            \item \emph{Qualità di processo - Materiale didattico del corso di Ingegneria del Software: https://www.math.unipd.it/~tullio/IS-1/2021/Dispense/T13.pdf};
            \item \emph{Qualità di prodotto - Materiale didattico del corso di Ingegneria del Software: https://www.math.unipd.it/~tullio/IS-1/2021/Dispense/T12.pdf};
            \end{itemize}
\section{Obiettivi e metriche di qualità}
    \subsection{Obiettivi di qualità}
        \subsubsection{Obiettivi di qualità di processo}
        \subsubsection{Obiettivi di qualità di prodotto}
    \subsection{Metriche di qualità}
        \subsubsection{Metriche di qualità di processo}
        \subsubsection{Metriche di qualità di prodotto}
\section{Specifica dei test}
\section{Test di unità}
\section{Test di integrazione}
\section{Test di sistema}
    \subsection{Test di sistema - Tracciamento dei requisiti}
\section{Test di accettazione}
\section{Resoconto attività di verifica}
\subsection{Periodo di analisi}
\section{Periodo di produzione del Proof of Concept}
