\renewcommand{\o}{Obbligatorio}
\renewcommand{\d}{Desiderabile}
\newcommand{\op}{Opzionale}
\section{Requisiti}
    Rappresentano dei requisiti che deve soddisfare il prodotto che si vuole realizzare.
    I requisiti saranno organizzati in forma tabellare.
    La tabella avrà le seguenti tre colonne:
    \begin{itemize}
        \item Codice identificativo
        \item Classificazione
        \item Descrizione
        \item Fonti
    \end{itemize}
    Codice identificativo dei requisiti
    Ogni requisito sarà strutturato come segue:
    \begin{itemize}
        \item Codice identificativo: R[Importanza][Tipologia][Codice]
            \begin{itemize}
                \item Importanza: 
                    \begin{itemize}
                        \item 1: Requisito obbligatorio, ovvero irrinunciabile per almeno uno degli stakeholder
                        \item 2: Requisito desiderabile, ovvero non strettamente necessario ma che porta valore aggiunto
                        riconoscibile
                        \item 3: Requisito opzionale, ovvero relativamente utile oppure contrattabile pi`u avanti nel progetto
                    \end{itemize}
                \item Tipologia:
                    \begin{itemize}
                        \item F: Funzionale, definisce una funzione di un sistema di uno o pi`u dei suoi componenti
                        \item Q: Qualitativo, definisce un requisito per garantire la qualit`a per un certo aspetto del progetto
                        \item P: Prestazionale, definisce un requisito che garantisce efficienza prestazionale nel prodotto
                        \item V: Vincolo, definisce un requisito volto a far rispettare un dato vincolo
                    \end{itemize}
                \item Codice:
                    Composto da:
                    \begin{itemize}
                        \item A* se il requisito proviene da un caso d'uso dell'applicazione web di gestione, * sarà composto da il
                        numero del caso d'uso.
                        \item C* se il requisito viene dal capitolato, * sarà un numero progressivo univoco.
                        \item I* se il requisito proviene da una decisione interna al gruppo, * sarà un numero progressivo univoco.
                        \item Z* se il requisito proviene da un incrontro con l'azienda.

                    \end{itemize}                    
            \end{itemize}
        \item Descrizione: descrizione sintetica del requisito.
        \item Classificazione: identifica l'importanza del requisito.
        \item Fonti: descrive da dove deriva il requisito. Le fonti sono le seguenti:
            \begin{itemize}
                \item Capitolato: Requisito individuato dalla lettura e/o analisi del capitolato.
                \item Interno: Requisito individuato dagli analisi quando hanno analizzato il problema.
                \item Casi d'uso: Requisito derivato da casi d'uso.
                \item Verbale: Requisito individuato durante incontri con azienda.
            \end{itemize}
    \end{itemize}
\subsection{Requisiti funzionali}
    {
        \definecolor{lightgray}{RGB}{234, 234, 234}
        \definecolor{darkblue}{RGB}{59,77,95}
        \rowcolors{2}{lightgray}{white}
        \renewcommand{\arraystretch}{1.5}
        
        \begin{longtable}{ l p{7cm} l p{3cm}}
            \caption{Tabella dei Requisiti funzionali}\\
            \rowcolor{darkblue}
            \textcolor{white}{Identificativo} & \textcolor{white}{Descrizione} & \textcolor{white}{Classificazione} & \textcolor{white}{Fonti}\\	
            \endfirsthead
            \rowcolor{darkblue}
            \textcolor{white}{Identificativo} & \textcolor{white}{Descrizione} & \textcolor{white}{Classificazione} & \textcolor{white}{Fonti}\\
            \endhead
            R1FI1 & Un utente non autenticato non può effettuare alcuna azione a meno di autenticazione e registrazione. & \o & Interno\\
            R1FA1.1& L'utente non registrato riceve via email l'invito a registrarsi & \o & UC1.1\\
            R1FA1.1.* & L'utente in fase di registrazione deve inserire email, nome, cognome e password & \o & UC1.1.1 UC1.1.2 UC1.1.3 UC1.1.4 UC1.1.5 Interno\\
            R1FA1.2 & L'utente registrato ma non autenticato deve essere in grado di accedere al servizio & \o & UC1.2 Interno\\
            R1FI2 & L'utente in fase di autenticazione inserisce username e/o password errate l'autenticazione fallisce e viene visualizzato un messaggio d'errore & \o & Interno UCE5\\
            R1FA1.3 & L'utente registrato non autenticato deve essere in grado di effettuare il reset della password qualora se la fosse dimenticata & \o & UC1.3 Interno\\
            R1FA2 & L'utente una volta effetuato il login deve poter accedere alla propria area personale & \o & UC2 Interno\\
            R1FA2.1 & L'utente autenticato deve poter eseguire il logout dal proprio account & \o & UC2.1 Interno\\
            R2FA2.2 & L'utente autenticato deve poter cambiare password del proprio account & \d & UC2.2 Interno\\
            R1FA3 & L'utente autenticato deve poter inserire una nuova traduzione & \o & UC3\\
            R1FA4 & L'utente autenticato admin/super admin deve poter revisionare una traduzione & \o & UC4\\
            R1FA4.2 & L'utente autenticato admin/super admin deve poter segnare in stato "da modificare" una traduzione errata & \o & UC4.2\\
            R1FA4.1 & L'utente autenticato admin/super admin deve poter pubblicare una traduzione & \o & UC4.1\\
            R1FA5 & L'utente autenticato deve poter modificare una traduzione & \o & UC5\\
            R1FA6 & L'utente autenticato admin/super admin deve poter eliminare una traduzione & \o & UC6\\
            R1FA7 & L'utente autenticato vuole visualizzare l'ultima modifica di una traduzione e quale utente l'ha
            effetuata e quando & \o & UC7\\
            R3FZ1 & L'utente autenticato vuole visualizzare lo storico delle modifiche di una traduzione e quale utente le ha effetuata e quando & \op & Incontro azienda\\
            R1FA8 & L'utente autenticato vuole ricercare una traduzione & \o & UC8\\
            R1FA8.1 & L'utente autenticato vuole ricercare una traduzione per id & \o & UC8.1\\
            R1FA8.2 & L'utente autenticato vuole ricercare una traduzione per parola & \o & UC8.2\\
            R2FA8 & L'utente autenticato vuole filtrare la ricerca & \d & UC8 Interno\\
            R2FA8.1 & L'utente autenticato vuole filtrare la ricerca per data creazione & \d & UC8.1 Interno\\
            R2FA8.2 & L'utente autenticato vuole filtrare la ricerca per pubblicazione & \d & UC8.2 Interno\\
            R2FA8.3 & L'utente autenticato vuole filtrare la ricerca per "da modificare" & \d & UC8.3 Interno\\
            R1FA10 & L'utente autenticato super admin vuole aggiungere un nuovo tenant & \o & UC10\\
            R1FA10.* & L'utente autenticato super admin alla creazione di un nuovo tenant vuole insereri il nome, il numero delle traduzioni disponibili, le lingue disponibili, la lingua di default da mostrare nella dashboard & \o & UC10.1 UC10.2 UC10.3 UC10.4\\
            R1FA11 & L'utente autenticato super admin vuole aggiungere una nuova lingua al tenant & \o & UC11\\
            R1FA12 & L'utente autenticato super admin vuole eliminare un tenant & \o & UC12\\
            R1FA13 & La libreria cliente vuole richiedere una traduzione & \o & UC13\\           
            
        \end{longtable}
        
    }


\subsection{Requisiti di vincolo}
        \definecolor{lightgray}{RGB}{234, 234, 234}
        \definecolor{darkblue}{RGB}{59,77,95}
        \rowcolors{2}{lightgray}{white}
        \renewcommand{\arraystretch}{1.5}
        
        \begin{longtable}{ l p{7cm} l p{3cm}}
            \caption{Tabella dei requisiti vincolo}\\
            \rowcolor{darkblue}
            \textcolor{white}{Identificativo} & \textcolor{white}{Descrizione} & \textcolor{white}{Classificazione} & \textcolor{white}{Fonti}\\	
            \endfirsthead
            \rowcolor{darkblue}
            \textcolor{white}{Identificativo} & \textcolor{white}{Descrizione} & \textcolor{white}{Classificazione} & \textcolor{white}{Fonti}\\
            \endhead
            R1VC1 & Il portale backoffice di gestione traduzioni dovrà funzionare su browser moderno & \o & capitolato\\
            R1VC2 & Il portale backoffice di gestione traduzioni dovrà garantire un supporto multi-tenant & \o & capitolato\\
            R1VC3 & Le API devono essere sviluppate in NodeJS & \o & capitolato\\
            R1VC4 & L'archittetura deve essere basata su micro-servizi & \o & capitolato\\
            R1VC5 & Il sistema dovrà sfruttare i servizi offerti da AWS & \o & capitolato\\
            R1VC6 & Il sistema dovrà sfruttare i servizi AWS Cognito & \o & capitolato\\
            R1VC7 & Il sistema dovrà sfruttare i servizi AWS DynamoDB & \o & capitolato\\
            R1VC8 & Il sistema dovrà sfruttare i servizi di calcolo AWS Lambda & \o & capitolato\\
            R1VC9 & Il sistema dovrà sfruttare i servizi AWS API Gateway & \o & capitolato\\
            R1VC10 & Il sistema dovrà sfruttare i servizi AWS Aurora Serverless & \o & capitolato\\
            %! react
        \end{longtable}
\subsection{Requisiti di qualità}
    \definecolor{lightgray}{RGB}{234, 234, 234}
    \definecolor{darkblue}{RGB}{59,77,95}
    \rowcolors{2}{lightgray}{white}
    \renewcommand{\arraystretch}{1.5}

    \begin{longtable}{ l p{7cm} l p{3cm}}
        \caption{Tabella dei requisiti qualità}\\
        \rowcolor{darkblue}
        \textcolor{white}{Identificativo} & \textcolor{white}{Descrizione} & \textcolor{white}{Classificazione} & \textcolor{white}{Fonti}\\	
        \endfirsthead
        \rowcolor{darkblue}
        \endhead
        R1QI1 & Le componenti software del sistema devono essere caricate in una repository di Github & \o & interno\\
        R1QC1 & Nel software dev'essere incluso un piano di test di unità & \o & capitolato\\
        R1QI2 & Il software deve avere un code coverage non inferiore all'80\% & \o & interno\\
        R1QC2 & Assieme alle API dev'esserci una documentazione dettagliata in formato swagger & \o & capitolato\\
        R1QC3 & Devono essere prodotti i diagrammi UML relativi agli use cases di progetto & \o & capitolato\\
        R1QC4 & Deve essere prodotto lo schema del design relativo al database & \o & capitolato\\
        R1QI3 & Creare il manuale utente per l'utilizzo della webapp & \o & interno\\
        R1QI4 & Creare il manuale per l'amministratore della webapp & \o & interno\\

    \end{longtable}

\subsection{Requisiti prestazionali}
    Non sono stati individuati requisiti prestazionali per quanto riguarda i requisiti obbligatori. I servizi di AWS utilizzati per realizzare la piattaforma non presentano problemi a livello di performance.
\subsection{Tracciamento requisiti}