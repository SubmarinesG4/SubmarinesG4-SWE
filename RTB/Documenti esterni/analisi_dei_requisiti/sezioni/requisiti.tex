\section{Requisiti}
    Rappresentano dei requisiti che deve soddisfare il prodotto che si vuole realizzare.
    I requisiti saranno organizzati in forma tabellare.
    La tabella avrà le seguenti tre colonne:
    \begin{itemize}
        \item Codice identificativo
        \item Classificazione
        \item Descrizione
        \item Fonti
    \end{itemize}
    Codice identificativo dei requisiti
    Ogni requisito sarà strutturato come segue:
    \begin{itemize}
        \item Codice identificativo: R[Importanza][Tipologia][Codice]
            \begin{itemize}
                \item Importanza: 
                    \begin{itemize}
                        \item 1: Requisito obbligatorio, ovvero irrinunciabile per almeno uno degli stakeholder
                        \item 2: Requisito desiderabile, ovvero non strettamente necessario ma che porta valore aggiunto
                        riconoscibile
                        \item 3: Requisito opzionale, ovvero relativamente utile oppure contrattabile pi`u avanti nel progetto
                    \end{itemize}
                \item Tipologia:
                    \begin{itemize}
                        \item F: Funzionale, definisce una funzione di un sistema di uno o pi`u dei suoi componenti
                        \item Q: Qualitativo, definisce un requisito per garantire la qualit`a per un certo aspetto del progetto
                        \item P: Prestazionale, definisce un requisito che garantisce efficienza prestazionale nel prodotto
                        \item V: Vincolo, definisce un requisito volto a far rispettare un dato vincolo
                    \end{itemize}
                \item Codice:
            \end{itemize}
    \end{itemize}
\subsection{Requisiti di vincolo}
\subsection{Requisiti di qualità}
\subsection{Requisiti funzionali}
\subsection{Tracciamento requisiti}