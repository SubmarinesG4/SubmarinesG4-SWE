\section{Casi d'uso}
\subsection{Scopo}
In questa sezione verranno presentati gli use cases tracciati, che il software dovrà implementare.
\subsection{UC 1 - Accesso}
    \begin{itemize}
        \item Attore primario: utente non autenticato.
        \item Precondizione: l'utente non è autenticato.
        \item Postcondizione: l'utente si registra [UC 1.1] o se è già in possesso delle credenziali effettua l'autenticazione [UC 1.2].
        \item Scenario principale: se l'utente non ha ancora un account dev'essere in possesso di un link di registrazione fornito dall'admin del proprio gruppo [UC 1.1], altrimenti può autenticarsi [UC 1.2].
        \item Scenario alternativo: l'utente già in possesso di un account ha scordato la password e ha bisogno di cambiarla in modo da poter accedere di nuovo [UC 1.3].
    \end{itemize}
    \subsubsection{UC 1.1 - Registrazione}
        \begin{itemize}
            \item Attore primario: utente non autenticato.
            \item Precondizione: l'utente non è registrato.
            \item Postcondizione: l'utente è registrato.
            \item Scenario principale: l'utente ha ricevuto una mail dall'admin del proprio gruppo per potersi e registrare e ha necessità di registrarsi.
            \item Flusso di eventi:
                \begin{enumerate}
                    \item L'utente inserisce il suo indirizzo email [UC 1.1.1].
                    \item L'utente inserisce il suo cognome [UC 1.1.2].
                    \item L'utente inserisce il suo nome [UC 1.1.3].
                    \item L'utente inserisce una password [UC 1.1.4].
                    \item L'utente reinserisce la password [UC 1.1.5].
                \end{enumerate}
        \end{itemize}
        \paragraph{UC 1.1.1 - Inserimento indirizzo email}
            \begin{itemize}
                \item Attore primario: utente non autenticato.
                \item Precondizione: l'utente non è registrato.
                \item Postcondizione: l'utente ha inserito l'indirizzo email.
                \item Scenario principale: l'utente inserisce l'indirizzo email per effettuare la registrazione.
                \item Estensioni:
                    \begin{itemize}
                        \item se l'indirizzo email è già nel sistema viene visualizzato un messaggio d'errore e la registrazione non viene eseguita [UCE 4].
                        \item se l'indirizzo email risulta una stringa vuota viene visualizzato un messaggio d'errore e la registrazione non viene eseguita [UCE 1].
                    \end{itemize}
            \end{itemize}
        \paragraph{UC 1.1.2 - Inserimento cognome}
            \begin{itemize}
                \item Attore primario: utente non autenticato.
                \item Precondizione: l'utente non è registrato.
                \item Postcondizione: l'utente ha inserito il cognome.
                \item Scenario principale: l'utente inserisce il cognome per effettuare la registrazione.
                \item Estensioni:
                    \begin{itemize}
                        \item Se il cognome risulta una stringa vuota viene visualizzato un messaggio d'errore e la registrazione non viene eseguita [UCE 1].
                    \end{itemize}
            \end{itemize}
        \paragraph{UC 1.1.3 - Inserire nome}
            \begin{itemize}
                \item Attore primario: utente non autenticato.
                \item Precondizione: l'utente non è registrato.
                \item Postcondizione: l'utente ha inserito il nome.
                \item Scenario principale: l'utente inserisce il nome per effettuare la registrazione.
                \item Estensioni:
                    \begin{itemize}
                        \item se il nome risulta una stringa vuota viene visualizzato un messaggio d'errore e la registrazione non viene eseguita [UCE 1].
                    \end{itemize}
            \end{itemize}
        \paragraph{UC 1.1.4 - Inserimento password}
            \begin{itemize}
                \item Attore primario: utente non autenticato.
                \item Precondizione: l'utente non è registrato.
                \item Postcondizione: l'utente ha inserito una password.
                \item Scenario principale: l'utente inserisce una password per effettuare la registrazione.
                \item Estensioni:
                    \begin{itemize}
                        \item se la password non risulta conforme agli standard di sicurezza viene visualizzato un messaggio d'errore e la registrazione non viene eseguita [UCE 2].
                        \item se la password risulta una stringa vuota viene visualizzato un messaggio d'errore e la registrazione non viene eseguita [UCE 1].
                    \end{itemize}
            \end{itemize}
        \paragraph{UC 1.1.5 - Reinserimento password}
            \begin{itemize}
                \item Attore primario: utente non autenticato.
                \item Precondizione: l'utente non è registrato.
                \item Postcondizione: l'utente ha reinserito la password.
                \item Scenario principale: l'utente reinserisce la password inserita in [UC 1.1.4] per effettuare la registrazione.
                \item Estensioni:
                    \begin{itemize}
                        \item se la password non è uguale a quella inserita in [UC 1.1.4] viene visualizzato un messaggio d'errore e la registrazione non viene eseguita [UCE 3].
                    \end{itemize}
            \end{itemize}
    \subsubsection{UC 1.2 - Autenticazione}
        \begin{itemize}
            \item Attore primario: utente non autenticato.
            \item Precondizione: l'utente non è autenticato.
            \item Postcondizione: l'utente è autenticato.
            \item Scenario principale: l'utente inserisce email e password e accede al sistema autenticandosi.
            \item Scenario alternativo: l'utente inserisce una coppia email-password non presente nel sistema.
            \item Flusso di eventi:
                \begin{enumerate}
                    \item L'utente inserisce il suo indirizzo email [UC 1.2.1].
                    \item L'utente inserisce la sua password [UC 1.2.2].
                \end{enumerate}
            \item Estensioni:
                \begin{itemize}
                    \item se la coppia email-password non è presente nel sistema l'autenticazione fallisce e viene visualizzato un messaggio d'errore [UCE 5].
                \end{itemize}
        \end{itemize}
        \paragraph{UC 1.2.1 - Inserimento indirizzo email}
            \begin{itemize}
                \item Attore primario: utente non autenticato.
                \item Precondizione: l'utente non è autenticato.
                \item Postcondizione: l'utente ha inserito l'indirizzo email.
                \item Scenario principale: l'utente inserisce l'indirizzo email per autenticarsi.
                \item Estensioni:
                    \begin{itemize}
                        \item se l'indirizzo email risulta una stringa vuota viene visualizzato un messaggio d'errore e l'autenticazione fallisce [UCE 1].
                    \end{itemize}
            \end{itemize}
        \paragraph{UC 1.2.2 - Inserimento password}
            \begin{itemize}
                \item Attore primario: utente non autenticato.
                \item Precondizione: l'utente non è autenticato.
                \item Postcondizione: l'utente ha inserito la password.
                \item Scenario principale: l'utente inserisce la password per autenticarsi.
                \item Estensioni:
                    \begin{itemize}
                        \item se la password risulta una stringa vuota viene visualizzato un messaggio d'errore e l'autenticazione fallisce [UCE 1].
                    \end{itemize}
            \end{itemize}
    \subsubsection{UC 1.3 - Password dimenticata}
        \begin{itemize}
            \item Attore primario: utente non autenticato.
            \item Precondizione: l'utente non è autenticato.
            \item Postcondizione: l'utente non è autenticato ma ha cambiato la password.
            \item Scenario principale: l'utente ha dimenticato la password e fa richiesta per cambiarla.
            \item Flusso di eventi:
                \begin{enumerate}
                    \item L'utente inserisce l'indirizzo email del proprio account.
                    \item Il sistema invia un email all'indirizzo inserito con il link alla pagina per il cambio della password.
                    \item L'utente inserisce una nuova password.
                    \item L'utente reinserisce la nuova password per confermarla.
                \end{enumerate}
        \end{itemize}
        \paragraph{UC 1.3.1 - Inserimento indirizzo email per il reset}
            \begin{itemize}
                \item Attore primario: utente non autenticato.
                \item Precondizione: l'utente non è autenticato.
                \item Postcondizione: l'utente ha inserito l'indirizzo email per il reset.
                \item Scenario principale: l'utente inserisce l'indirizzo email per farsi mandare il link di cambio password.
                \item Estensioni:
                    \begin{itemize}
                        \item se l'indirizzo email risulta una stringa vuota viene visualizzato un messaggio d'errore e la mail non viene inviata [UCE 1].
                        \item se l'utente inserisce un'indirizzo email non presente nel sistema non riceverà nessuna email.
                    \end{itemize}
            \end{itemize}
        \paragraph{UC 1.3.2 - Inserimento nuova password}
            \begin{itemize}
                \item Attore primario: utente non autenticato.
                \item Precondizione: l'utente non è autenticato.
                \item Postcondizione: l'utente ha inserito la password.
                \item Scenario principale: l'utente inserisce la password nella pagina corrispondente al link arrivato via mail.
                \item Estensioni:
                    \begin{itemize}
                        \item se la password non risulta conforme agli standard di sicurezza viene visualizzato un messaggio d'errore e il reset non viene eseguito [UCE 2].
                        \item se la password risulta una stringa vuota viene visualizzato un messaggio d'errore e il cambio password non viene eseguito [UCE 1].
                    \end{itemize}
            \end{itemize}
        \paragraph{UC 1.3.3 - Reinserimento password}
            \begin{itemize}
                \item Attore primario: utente non autenticato.
                \item Precondizione: l'utente non è autenticato.
                \item Postcondizione: l'utente ha reinserito la password.
                \item Scenario principale: l'utente reinserisce la password nella pagina corrispondente al link arrivato via mail.
                \item Estensioni:
                    \begin{itemize}
                        \item se la password non è uguale a quella inserita in [UC 1.3.2] viene visualizzato un messaggio d'errore e il reset non viene eseguito [UCE 3].
                    \end{itemize}
            \end{itemize}
\subsection{UC 2 - Area personale}
    \begin{itemize}
        \item Attore primario: utente traduttore/utente admin/utente super admin(zero12).
        \item Precondizione: l'utente è autenticato.
        \item Postcondizione: l'utente accede alla propria area personale.
        \item Scenario principale: l'utente accede alla sua area personale per gestire il proprio profilo.
        \item Sottocasi:
            \begin{itemize}
                \item L'utente vuole uscire dal proprio account [UC 2.1].
                \item L'utente vuole cambiare la propria password [UC 2.2].
            \end{itemize}
    \end{itemize}
    \subsubsection{UC 2.1 - Logout}
        \begin{itemize}
            \item Attore primario: utente traduttore/utente admin/utente super admin(zero12).
            \item Precondizione: l'utente è autenticato.
            \item Postcondizione: l'utente non è più autenticato.
            \item Scenario principale: l'utente vuole uscire dal proprio account.
        \end{itemize}
    \subsubsection{UC 2.2 - Cambio password}
        \begin{itemize}
            \item Attore primario: utente autenticato.
            \item Precondizione: l'utente vuole cambiare la propria password.
            \item Postcondizione: l'utente ha cambiato password.
            \item Scenario principale: l'utente ha neccistà di cambiare la propria password, in genere quando questa non è più sicura.
            \item Flusso di eventi:
                \begin{enumerate}
                    \item L'utente inserisce la vecchia password per riconfermare la sua identità [UC 2.2.1].
                    \item L'utente inserisce la nuova password [UC 2.2.2].
                    \item Per sicurezza il sistema chiede il reinserimento della nuova password [UC 2.2.3].
                \end{enumerate}
        \end{itemize}
        \paragraph{UC 2.2.1 - Inserimento password vecchia}
            \begin{itemize}
                \item Attore primario: utente autenticato.
                \item Precondizione: l'utente ha necessità di cambiare password.
                \item Postcondizione: l'utente ha inserito la vecchia password.
                \item Scenario principale: inserimento vecchia password per modificarla.
                \item Estensioni:
                    \begin{itemize}
                        \item se la password è sbagliata viene visualizzato un messaggio d'errore [UCE 6].
                        \item se la password risulta una stringa vuota viene visualizzato un messaggio d'errore e il cambio password non viene eseguito [UCE 1].
                    \end{itemize}
            \end{itemize}
        \paragraph{UC 2.2.2 - Inserimento password nuova}
            \begin{itemize}
                \item Attore primario: utente autenticato.
                \item Precondizione: l'utente ha necessità di cambiare password ed ha inserito la vecchia password.
                \item Postcondizione: l'utente non autenticato ha inserito la nuova password.
                \item Scenario principale: inserimento nuova password.
                \item Estensioni:
                    \begin{itemize}
                        \item se la password non è conforme agli standard di sicurezza viene visualizzato un messaggio d'errore [UCE 2].
                        \item se la password risulta una stringa vuota viene visualizzato un messaggio d'errore e il cambio password non viene eseguito [UCE 1].
                    \end{itemize}
            \end{itemize}
        \paragraph{UC 2.2.3 - Conferma password}
            \begin{itemize}
                \item Attore primario: utente autenticato.
                \item Precondizione: l'utente ha necessità di cambiare password ed ha inserito la vecchia e la nuova password.
                \item Postcondizione: l'utente non autenticato ha inserito di nuovo la nuova password.
                \item Scenario principale: conferma nuova password per evitare errori.
                \item Estensioni:
                    \begin{itemize}
                        \item se la password non è uguale a quella inserita in precedenza viene visualizzato un messaggio d'errore [UCE 3].
                    \end{itemize}
            \end{itemize}
\subsection{UC 3 - Inserimento nuova traduzione}
    \begin{itemize}
        \item Attore primario: utente traduttore/utente admin/utente super admin(zero12).
        \item Precondizione: l'utente vuole inserire una nuova traduzione.
        \item Postcondizione: l'utente ha inserito una nuova traduzione.
        \item Scenario principale: l'utente clicca sul pulsante "Nuova traduzione", compila i campi dati relativi e clicca il pulsante "Conferma".
        \item Flusso di eventi:
            \begin{enumerate}
                \item L'utente seleziona il progetto.
                \item L'utente clicca sul pulsante "Nuova traduzione".
                \item L'utente scrive la frase/parola da tradurre e la relativa traduzione.
                \item L'utente clicca sul pulsante "Salva".
            \end{enumerate}
        \item Estensioni:
            \begin{itemize}
                \item se l'utente non inserisce almeno traduzioni in due lingue diverse viene visualizzato un messaggio d'errore e la creazione fallisce [UCE 7].
            \end{itemize}
    \end{itemize}
\subsection{UC 4 - Revisione traduzione}
    \begin{itemize}
        \item Attore primario: utente admin/utente super admin(zero12).
        \item Precondizione: l'utente vuole revisionare una traduzione.
        \item Postcondizione: l'utente ha revisionato la traduzione.
        \item Scenario principale: l'utente vuole revisionare una traduzione e la approva [UC 4.1].
        \item Scenario alternativo: l'utente vuole revisionare una traduzione e la mette in "modifica" visto che è sbagliata [UC 4.2].
        \item Flusso di eventi:
            \begin{enumerate}
                \item L'utente seleziona la traduzione che vuole revisionare e clicca sul pulsante "Revisiona".
                \item Sottocasi:
                    \begin{enumerate}
                        \item L'utente pubblica la traduzione [UC 4.1].
                        \item L'utente segnala che la traduzione è da modificare [UC 4.2].
                    \end{enumerate}
            \end{enumerate}
    \end{itemize}
    \subsubsection{UC 4.1 - Pubblicazione}
        \begin{itemize}
            \item Attore primario: utente admin/utente super admin(zero12).
            \item Precondizione: l'utente ha revisionato una traduzione e la pubblica.
            \item Postcondizione: l'utente ha pubblicato la traduzione.
            \item Scenario principale: l'utente vuole pubblicare una traduzione.
        \end{itemize}
    \subsubsection{UC 4.2 - Traduzione da modificare}
        \begin{itemize}
            \item Attore primario: utente admin/utente super admin(zero12).
            \item Precondizione: l'utente vuole segnalare che la traduzione è da modificare.
            \item Postcondizione: l'utente segnala che la traduzione è da modificare.
            \item Scenario principale: l'utente vuole segnalare la traduzione da modificare.
        \end{itemize}
\subsection{UC 5 - Modifica traduzione}
    \begin{itemize}
        \item Attore primario: utente traduttore/utente admin/utente super admin(zero12).
        \item Precondizione: l'utente vuole modificare una traduzione.
        \item Postcondizione: l'utente ha modificato la traduzione.
        \item Scenario principale: l'utente vuole modificare una traduzione.
        \item Flusso di eventi:
            \begin{enumerate}
                \item L'utente seleziona la traduzione che vuole modificare.
                \item L'utente seleziona scorre tra le lingue in cui è tradotta la frase/parola e sceglie quale vuole modificare.
                \item L'utente modifica la traduzione.
                \item L'utente clicca sul pulsante "Salva".
            \end{enumerate}
    \end{itemize}
\subsection{UC 6 - Elimina traduzione}
    \begin{itemize}
        \item Attore primario: utente admin/utente super admin(zero12).
        \item Precondizione: l'utente vuole eliminare una traduzione.
        \item Postcondizione: l'utente ha eliminato la traduzione.
        \item Scenario principale: l'utente elimina una traduzione.
        \item Flusso di eventi:
            \begin{enumerate}
                \item L'utente seleziona la traduzione che vuole eliminare.
                \item L'utente clicca sul tasto "Elimina".
                \item Viene visualizzato un messaggio per confermare l'eliminazione.
                \item L'utente clicca sul tasto di conferma eliminazione.
            \end{enumerate}
    \end{itemize}
\subsection{UC 7 - Versionamento}
    \begin{itemize}
        \item Attore primario: utente traduttore/utente admin/utente super admin(zero12).
        \item Precondizione: l'utente vuole visualizzare le ultime N modifiche di una traduzione, quale utente le ha effetuate e quando.
        \item Postcondizione: l'utente visualizza le ultime N modifica di una traduzione, quale utente le ha effetuate e quando.
        \item Scenario principale: l'utente ha visualizzato il versionamento.
        \item Flusso di eventi:
        \begin{enumerate}
            \item L'utente seleziona la traduzione sulla quale vuole visualizzare le informazioni sulle ultime modifiche.
            \item L'utente clicca sul tasto "Ultime modifiche".
            \item Vengono visualizzate le informazioni sulle ultime modifiche.
        \end{enumerate}
    \end{itemize}   
\subsection{UC 8 - Ricerca}
    \begin{itemize}
        \item Attore primario: utente traduttore/utente admin/utente super admin(zero12).
        \item Precondizione: l'utente all'interno di un progetto ha necessità di ricercare una traduzione.
        \item Postcondizione: l'utente ha come risultato della ricerca tutte le traduzioni che rispettano la query inserita.
        \item Scenario principale: l'utente vuole ricercare una traduzione.
        \item Flusso di eventi:
            \begin{enumerate}
                \item L'utente inserisce l'id o parola nel campo di ricerca.
                \item Il sistema ritorna tutte le traduzioni che rispettano la query inserita.
            \end{enumerate}
        \item Sottocasi:
                \begin{enumerate}
                    \item L'utente ricerca per id [UC 7.1].
                    \item L'utente ricerca per parola [UC 7.2].
                \end{enumerate}
    \end{itemize}
    \subsubsection{UC 8.1 - Ricerca per id univoco}%??????
        \begin{itemize}
            \item Attore primario: utente traduttore/utente admin/utente super admin(zero12).
            \item Precondizione: l'utente ha necessità trovare una traduzione con l'id.
            \item Postcondizione: l'utente ha come risultato della ricerca tutte le traduzioni che rispettano l'id anche parzialmente inserito.%????
            \item Scenario principale: l'utente inserisce nel campo di ricerca l'id univoco.
        \end{itemize}
    \subsubsection{UC 8.2 - Ricerca per parola}
        \begin{itemize}
            \item Attore primario: utente traduttore/utente admin/utente super admin(zero12).
            \item Precondizione: l'utente ha necessità trovare una traduzione data una o più parole.
            \item Postcondizione: l'utente ha come risultato della ricerca tutte le traduzioni che rispettano le parole inserite.
            \item Scenario principale: l'utente inserisce nel campo di ricerca l'id univoco.
        \end{itemize}
\subsection{UC 9 - Filtraggio ricerca} %! nice to have
    \begin{itemize}
        \item Attore primario: utente traduttore/utente admin/utente super admin(zero12).
        \item Precondizione: l'utente ha necessità di filtrare la ricerca effettuata.
        \item Postcondizione: l'utente ha come risultato della ricerca tutte le traduzioni che rispettano i filtri inseriti.
        \item Scenario principale: l'utente vuole filtrare una ricerca.
        \item Sottocasi:
            \begin{itemize}
                \item Filtraggio per data di creazione [UC 8.1]
                \item Filtraggio per pubblicazione [UC 8.2]
                \item Filtraggio per in "modifica" [UC 8.3]
            \end{itemize}
    \end{itemize}  
    \subsubsection{UC 9.1 - Filtraggio per data di creazione}
        \begin{itemize}
            \item Attore primario: utente traduttore/utente admin/utente super admin(zero12).
            \item Precondizione: l'utente ha necessità di filtrare la ricerca effettuata per la data di creazione.
            \item Postcondizione: l'utente ha come risultato della ricerca tutte le traduzioni che rispettano la data di creazione precedentemente selezionata. 
            \item Scenario principale: l'utente deve selezionare la data di creazione per eseguire la ricerca filtrata.
        \end{itemize}
    \subsubsection{UC 9.2 - Filtraggio per pubblicazione}
        \begin{itemize}
            \item Attore primario: utente traduttore/utente admin/utente super admin(zero12).
            \item Precondizione: l'utente ha necessità di filtrare la ricerca effetuata per pubblicazione.
            \item Postcondizione: l'utente ha come risultato della ricerca tutte le traduzioni che rispettano la pubblicazione precedentemente selezionata. 
            \item Scenario principale: l'utente deve selezionare lo stato della traduzione in "pubblicata" per eseguire la ricerca filtrata.
        \end{itemize}
    \subsubsection{UC 9.3 - Filtraggio per in "modifica"}
        \begin{itemize}
            \item Attore primario: utente traduttore/utente admin/utente super admin(zero12).
            \item Precondizione: l'utente ha necessità di filtrare la ricerca effettuata per traduzioni in "modifica".
            \item Postcondizione: l'utente ha come risultato della ricerca tutte le traduzioni che sono nella lista da "modificare". 
            \item Scenario principale: l'utente deve selezionare la stato della traduzione in "modifica" per eseguire la ricerca filtrata.
        \end{itemize}
\subsection{UC 10 - Aggiunta tenant}
    \begin{itemize}
        \item Attore primario: utente super admin(zero12).
        \item Precondizione: l'utente ha necessità di aggiungere un nuovo tenant.
        \item Postcondizione: l'utente ha aggiunto il tenant.
        \item Scenario principale: l'utente inserisce il nuovo tenant.
        \item Flusso di eventi:
        \begin{enumerate}
            \item L'utente inserisce il nome del nuovo tenant [UC 10.1].
            \item L'utente inserisce il numero di traduzioni disponibili [UC 10.2].
            \item L'utente inserisce le lingue disponibili nel nuovo tenant [UC 10.3].
            \item L'utente sceglie quale lingua mostrare di default nella dashboard del tenant [UC 10.4].
        \end{enumerate}
    \end{itemize}    
    \subsubsection{UC 10.1 - Inserimento nome tenant}
        \begin{itemize}
            \item Attore primario: utente super admin(zero12).
            \item Precondizione: l'utente ha necessità di inserire un tenant, la procedura di inserimento richiede il nome del progetto.
            \item Postcondizione: l'utente ha inserito il nome del tenant.
            \item Scenario principale: l'utente inserisce il nome del tenant.
            \item Estensioni: 
                \begin{itemize}
                    \item l'utente deve inserire obbligatoriamente il nome del tenant univoco con almeno un carattere altrimenti viene visualizzato un messaggio d'errore e la creazione fallisce [UCE 1].
                \end{itemize}
        \end{itemize}
    \subsubsection{UC 10.2 - Inserimento numero traduzioni disponibili}
        \begin{itemize}
            \item Attore primario: utente super admin(zero12).
            \item Precondizione: l'utente ha necessità di inserire il numero delle traduzioni disponibili(si chiede una fatturazione per lo spazio su disco richiesto per salvare il numero di traduzioni richieste).
            \item Postcondizione: l'utente ha inserito il numero delle traduzioni.
            \item Scenario principale: l'utente inscerisce il numero delle traduzioni a disponibilità del tenant.
            \item Estensioni: 
                \begin{itemize}
                    \item se il numero delle traduzioni disponibili inserito dall'utente è minore di 1 viene visualizzato un messaggio d'errore e la creazione del tenant fallisce [UCE 8].
                \end{itemize}
        \end{itemize}
    \subsubsection{UC 10.3 - Inserimento lingue disponibili}
        \begin{itemize}
            \item Attore primario: utente super admin(zero12).
            \item Precondizione: l'utente ha necessità di inserire le lingue disponibili nel tenant.
            \item Postcondizione: l'utente ha inserito le lingue disponibili nel tenant.
            \item Scenario principale: l'utente inscerisce le lingue disponibili nel tenant.
            \item Estensioni: 
                \begin{itemize}
                    \item se l'utente non inserisce almeno due lingue viene visualzzato un messaggio d'errore e la creazione del tenant fallisce [UCE 9].
                    \item se l'utente cerca di inserire una lingua che corrisponde a una stringa vuota viene visualizzato un messaggio d'errore e la creazione del tenant fallisce [UCE 1].
                \end{itemize}
        \end{itemize}
    \subsubsection{UC 10.4 - Selezione lingua di default}
        \begin{itemize}
            \item Attore primario: utente super admin(zero12).
            \item Precondizione: l'utente ha necessità di selezionare la lingua di default da visualizzare nella dashboard del tenant.
            \item Postcondizione: l'utente ha selezionato tra le lingue disponibili [UC 10.3] la lingua di default nella dashboard del tenant.
            \item Scenario principale: l'utente seleziona la lingua di default da visualizzare nella dashboard del tenant.
        \end{itemize}
\subsection{UC 11 Aggiunta lingua tenant}
    \begin{itemize}
        \item Attore primario: utente super admin(zero12).
        \item Precondizione: l'utente ha necessità di inserire una nuova la lingua nel tenant.
        \item Postcondizione: l'utente ha inserito la nuova lingua nel tenant.
        \item Scenario principale: l'utente inscerisce la nuova lingua nel tenant.
        \item Flusso di eventi:
            \begin{itemize}
                \item L'utente scrive nel campo di testo la nuova lingua.
                \item L'utente clicca su "Salva".
            \end{itemize}
        \item Estensioni:
            \begin{itemize}
                \item se l'utente lascia il campo della lingua vuoto viene visualizzato un messaggio d'errore e la nuova lingua non viene inserita [UCE 1].
                \item se la lingua è già presente nel tenant viene visualizzato un messaggio d'errore e la lingua non viene inserita [UCE 10].
            \end{itemize}
        \end{itemize}
\subsection{UC 12 - Eliminazione tenant}
    \begin{itemize}
        \item Attore primario: utente super admin(zero12).
        \item Precondizione: l'utente ha necessità di eliminare un tenant.
        \item Postcondizione: l'utente ha eliminato il tenant.
        \item Scenario principale: l'utente trova il tenant e lo elimina.
        \item Flusso di eventi:
            \begin{enumerate}
                \item l'utente visualizza la schermata dei tenant.
                \item l'utente trova quello che vuole eliminare.
                \item l'utente clicca sul pulsante "Elimina".
            \end{enumerate}
    \end{itemize}
\subsection{UC 13 - Ricerca traduzione da libreria client}
    \begin{itemize}
        \item Attore primario: libreria client.
        \item Precondizione: libreria client ha necessità di ricercare una traduzione.
        \item Postcondizione: libreria client ha ricevuto la traduzione.
        \item Scenario principale: libreria client per richiedere la traduzione deve autenticarsi all'API in modo tale da ricevere solo le traduzioni a cui ha accesso.
    \end{itemize}
\subsection{UCE 1 - Visualizzazione messaggio d'errore nel caso: campo di testo obbligatorio non riempito}
    \begin{itemize}
        \item Attore primario: qualsiasi utente.
        \item Precondizione: l'utente lascia vuoto un campo di testo che deve essere obbligatoriamente riempito.
        \item Postcondizione: viene visualizzato un messaggio d'errore e l'operazione che doveva essere eseguita viene abortita.
        \item Scenario principale: l'utente riceve il messaggio d'errore con l'indicazione di dove è stato, per poi poter ricompilare il campo di testo.
    \end{itemize}
\subsection{UCE 2 - Visualizzazione messaggio d'errore nel caso: password inserita non conforme agli standard di sicurezza minimi}
    \begin{itemize}
        \item Attore primario: qualsiasi utente.
        \item Precondizione: l'utente inserisce una password che non è conforme agli standard di sicurezza minimi richiesti dal sistema.
        \item Postcondizione: viene visualizzato un messaggio d'errore e l'operazione che doveva essere eseguita viene abortita.
        \item Scenario principale: l'utente riceve il messaggio d'errore con la spiegazione dei requisiti della password, in modo da poter inserire una password adatta.
    \end{itemize}
\subsection{UCE 3 - Visualizzazione messaggio d'errore nel caso: nella conferma della password non c'è corrispondenza}
    \begin{itemize}
        \item Attore primario: qualsiasi utente.
        \item Precondizione: nel reinserimento di sicurezza della password l'utente inserisce una password che è diversa da quella precedente.
        \item Postcondizione: viene visualizzato un messaggio d'errore e la creazione/cambio password viene abortita.
        \item Scenario principale: l'utente riceve il messaggio d'errore che spiega cosa è successo e si da la possibilità all'utente di completare di nuovo il form.
    \end{itemize}
\subsection{UCE 4 - Visualizzazione messaggio d'errore nel caso: indirizzo email già presente nel sistema in fase di registrazione}
    \begin{itemize}
        \item Attore primario: qualsiasi utente.
        \item Precondizione: durante la registrazione viene inserito un indirizzo email già presente nel sistema.
        \item Postcondizione: viene visualizzato un messaggio d'errore e la creazione dell'utente viene abortita.
        \item Scenario principale: l'utente riceve il messaggio d'errore che avvisa del problema e si da la possibilità all'utente di completare di nuovo il form di registrazione.
    \end{itemize}
\subsection{UCE 5 - Visualizzazione messaggio d'errore nel caso: account non presente nel sistema}
    \begin{itemize}
        \item Attore primario: qualsiasi utente.
        \item Precondizione: l'utente prova ad autenticarsi con delle credenziali non presenti nel sistema.
        \item Postcondizione: viene visualizzato un messaggio d'errore e l'utente non viene autenticato.
        \item Scenario principale: l'utente riceve il messaggio d'errore e gli viene data la possibilità di riprovare.
    \end{itemize}
\subsection{UCE 6 - Visualizzazione messaggio d'errore nel caso: in fase di cambio password la passoword da inserire per confermare l'identità è sbagliata}
    \begin{itemize}
        \item Attore primario: qualsiasi utente.
        \item Precondizione: l'utente inserisce una password sbagliata provando a confermare la sua identità per cambiare password.
        \item Postcondizione: viene visualizzato un messaggio d'errore, il cambio password non viene eseguito.
        \item Scenario principale: l'utente riceve il messaggio d'errore e ottiene una nuova possibilità di cambiare la password.
    \end{itemize}
\subsection{UCE 7 - Visualizzazione messaggio d'errore nel caso: nell'inserimento di una traduzione l'utente non inserisce almeno due lingue diverse}
    \begin{itemize}
        \item Attore primario: qualsiasi utente.
        \item Precondizione: durante la fase di inserimento di una traduzione l'utente ha inserito l'espressione/parola in una sola o in nessuna lingua.
        \item Postcondizione: viene visualizzato un messaggio d'errore che avvisa che bisogna inserire l'espressione/parola in almeno due lingue e l'inserimento non viene completato.
        \item Scenario principale: l'utente riceve il messaggio d'errore e può ricompilare il form d'inserimento della traduzione.
    \end{itemize}
\subsection{UCE 8 - Visualizzazione messaggio d'errore nel caso: nell'inserimento del numero di traduzioni disponibili l'utente inserisce un numero non corretto}
    \begin{itemize}
        \item Attore primario: utente super admin(zero12).
        \item Precondizione: l'utente in fase di creazione di un nuovo tenant inserisce un numero di traduzioni minore o uguale a 0.
        \item Postcondizione: la creazione del tenant fallisce e viene visualizzato un messaggio d'errore.
        \item Scenario principale: l'utente riceve il messaggio d'errore e ha la possibilità di ricompilare il form.
    \end{itemize}
\subsection{UCE 9 - Visualizzazione messaggio d'errore nel caso: nell'inserimento di un nuovo tenant l'utente non inserisce almeno due lingue}
    \begin{itemize}
        \item Attore primario: utente super admin(zero12).
        \item Precondizione: l'utente in fase di inserimento di un nuovo tenant inserisce meno di due lingue.
        \item Postcondizione: viene visualizzato un messaggio d'errore, l'inserimento del nuovo tenant fallisce.
        \item Scenario principale: l'utente riceve il messaggio d'errore e ottiene una nuova possibilità di compilare il form.
    \end{itemize}
\subsection{UCE 10 - Visualizzazione messaggio d'errore nel caso: inserimento doppio di una lingua in un tenant}
    \begin{itemize}
        \item Attore primario: qualsiasi utente.
        \item Precondizione: l'utente inserisce una lingua già presente nel tenant.
        \item Postcondizione: viene visualizzato un messaggio d'errore, non viene aggiunta nessuna lingua.
        \item Scenario principale: l'utente riceve il messaggio d'errore e ha la possibilità di poter reinserire il nome della lingua.
    \end{itemize}

%! UTENTI SECONDARI 