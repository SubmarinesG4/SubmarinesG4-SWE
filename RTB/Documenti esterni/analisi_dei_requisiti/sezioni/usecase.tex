\section{Casi d'uso}
\subsection{Scopo}
In questa sezione verranno presentati gli use cases tracciati, che il software dovrà implementare.
\subsection{UC 1 - Accesso}
    \begin{itemize}
        \item Attore primario: utente non autenticato.
        \item Precondizione: l'utente non è autenticato.
        \item Postcondizione: l'utente si registra [UC 1.1] o se è già in possesso delle credenziali effettua l'autenticazione [UC 1.2].
        \item Scenario principale: se l'utente non ha ancora un account dev'essere in possesso di un link di registrazione fornito dall'admin del proprio gruppo [UC 1.1], altrimenti può autenticarsi [UC 1.2].
        \item Scenario alternativo: l'utente già in possesso di un account ha scordato la password e ha bisogno di cambiarla in modo da poter accedere di nuovo [UC 1.3].
    \end{itemize}
    \subsubsection{UC 1.1 - Registrazione}
        \begin{itemize}
            \item Attore primario: utente non autenticato.
            \item Precondizione: l'utente non è registrato.
            \item Postcondizione: l'utente è registrato.
            \item Scenario principale: l'utente ha ricevuto una mail dall'admin del proprio gruppo per potersi e registrare e ha necessità di registrarsi.
            \item Flusso di eventi:
                \begin{enumerate}
                    \item L'utente inserisce il suo indirizzo email [UC 1.1.1].
                    \item L'utente inserisce il suo cognome [UC 1.1.2].
                    \item L'utente inserisce il suo nome [UC 1.1.3].
                    \item L'utente inserisce una password [UC 1.1.4].
                    \item L'utente reinserisce la password [UC 1.1.5].
                \end{enumerate}
        \end{itemize}
        \paragraph{UC 1.1.1 - Inserimento indirizzo email}
            \begin{itemize}
                \item Attore primario: utente non autenticato.
                \item Precondizione: l'utente non è registrato.
                \item Postcondizione: l'utente ha inserito l'indirizzo email.
                \item Scenario principale: l'utente inserisce l'indirizzo email per effettuare la registrazione.
                \item Estensioni:
                    \begin{itemize}
                        \item Se l'indirizzo email è già stato inserito viene visualizzato un messaggio d'errore e la registrazione non viene eseguita.
                    \end{itemize}
            \end{itemize}
        \paragraph{UC 1.1.2 - Inserimento cognome}
            \begin{itemize}
                \item Attore primario: utente non autenticato.
                \item Precondizione: l'utente non è registrato.
                \item Postcondizione: l'utente ha inserito il cognome.
                \item Scenario principale: l'utente inserisce il cognome per effettuare la registrazione.
                \item Estensioni:
                    \begin{itemize}
                        \item Se il cognome risulta una stringa vuota viene visualizzato un messaggio d'errore e la registrazione non viene eseguita.
                    \end{itemize}
            \end{itemize}
        \paragraph{UC 1.1.3 - Inserire nome}
            \begin{itemize}
                \item Attore primario: utente non autenticato.
                \item Precondizione: l'utente non è registrato.
                \item Postcondizione: l'utente ha inserito il nome.
                \item Scenario principale: l'utente inserisce il nome per effettuare la registrazione.
                \item Estensioni:
                    \begin{itemize}
                        \item Se il nome risulta una stringa vuota viene visualizzato un messaggio d'errore e la registrazione non viene eseguita.
                    \end{itemize}
            \end{itemize}
        \paragraph{UC 1.1.4 - Inserimento password}
            \begin{itemize}
                \item Attore primario: utente non autenticato.
                \item Precondizione: l'utente non è registrato.
                \item Postcondizione: l'utente ha inserito una password.
                \item Scenario principale: l'utente inserisce una password per effettuare la registrazione.
                \item Estensioni:
                    \begin{itemize}
                        \item Se la password non risulta conforme agli standard di sicurezza viene visualizzato un messaggio d'errore e la registrazione non viene eseguita.
                    \end{itemize}
            \end{itemize}
        \paragraph{UC 1.1.5 - Reinserimento password}
            \begin{itemize}
                \item Attore primario: utente non autenticato.
                \item Precondizione: l'utente non è registrato.
                \item Postcondizione: l'utente ha reinserito la password.
                \item Scenario principale: l'utente reinserisce la password inserita in [UC 1.1.4] per effettuare la registrazione.
                \item Estensioni:
                    \begin{itemize}
                        \item Se la password non è uguale a quella inserita in [UC 1.1.4] viene visualizzato un messaggio d'errore e la registrazione non viene eseguita.
                    \end{itemize}
            \end{itemize}
    \subsubsection{UC 1.2 - Autenticazione}
        \begin{itemize}
            \item Attore primario: utente non autenticato.
            \item Precondizione: l'utente non è autenticato.
            \item Postcondizione: l'utente è autenticato.
            \item Scenario principale: l'utente inserisce email e password e accede al sistema autenticandosi.
            \item Scenario alternativo: l'utente inserisce una coppia email-password non presente nel sistema.
            \item Flusso di eventi:
                \begin{enumerate}
                    \item L'utente inserisce il suo indirizzo email [UC 1.2.1].
                    \item L'utente inserisce la sua password [UC 1.2.2].
                \end{enumerate}
            \item Estensioni:
                \begin{itemize}
                    \item se la coppia email-password non è presente nel sistema l'autenticazione fallisce e viene visualizzato un messaggio d'errore.
                \end{itemize}
        \end{itemize}
        \paragraph{UC 1.2.1 - Inserimento indirizzo email}
            \begin{itemize}
                \item Attore primario: utente non autenticato.
                \item Precondizione: l'utente non è autenticato.
                \item Postcondizione: l'utente ha inserito l'indirizzo email.
                \item Scenario principale: l'utente inserisce l'indirizzo email per autenticarsi.
            \end{itemize}
        \paragraph{UC 1.2.2 - Inserimento password}
            \begin{itemize}
                \item Attore primario: utente non autenticato.
                \item Precondizione: l'utente non è autenticato.
                \item Postcondizione: l'utente ha inserito la password.
                \item Scenario principale: l'utente inserisce la password per autenticarsi.
            \end{itemize}
    \subsubsection{UC 1.3 - Password dimenticata}
        \begin{itemize}
            \item Attore primario: utente non autenticato.
            \item Precondizione: l'utente non è autenticato.
            \item Postcondizione: l'utente ha cambiato la password.
            \item Scenario principale: l'utente ha dimenticato la password e fa richiesta per cambiarla.
            \item Flusso di eventi:
                \begin{enumerate}
                    \item L'utente inserisce l'indirizzo email del proprio account.
                    \item Il sistema invia un email all'indirizzo inserito con il link alla pagina per il cambio della password.
                    \item L'utente inserisce una nuova password.
                    \item L'utente reinserisce la nuova password per confermarla.
                \end{enumerate}
            \item Estensioni:
                \begin{itemize}
                    \item Se l'utente inserisce un'indirizzo email non presente nel sistema non riceverà nessuna email.
                \end{itemize}
        \end{itemize}
        \paragraph{UC 1.3.1 - Inserimento indirizzo email per il reset}
            \begin{itemize}
                \item Attore primario: utente non autenticato.
                \item Precondizione: l'utente non è autenticato.
                \item Postcondizione: l'utente ha inserito l'indirizzo email per il reset.
                \item Scenario principale: l'utente inserisce l'indirizzo email per farsi mandare il link di cambio password.
            \end{itemize}
        \paragraph{UC 1.3.2 - Inserimento nuova password}
            \begin{itemize}
                \item Attore primario: utente non autenticato.
                \item Precondizione: l'utente non è autenticato.
                \item Postcondizione: l'utente ha inserito la password.
                \item Scenario principale: l'utente inserisce la password nella pagina corrispondente al link arrivato via mail.
                \item Estensioni:
                        \begin{itemize}
                            \item Se la password non risulta conforme agli standard di sicurezza viene visualizzato un messaggio d'errore e il reset non viene eseguito.
                        \end{itemize}
            \end{itemize}
        \paragraph{UC 1.3.3 - Reinserimento password}
            \begin{itemize}
                \item Attore primario: utente non autenticato.
                \item Precondizione: l'utente non è autenticato.
                \item Postcondizione: l'utente ha reinserito la password.
                \item Scenario principale: l'utente reinserisce la password nella pagina corrispondente al link arrivato via mail.
                \item Estensioni:
                        \begin{itemize}
                            \item Se la password non è uguale a quella inserita in [UC 1.3.2] viene visualizzato un messaggio d'errore e il reset non viene eseguito.
                        \end{itemize}
            \end{itemize}

\subsection{UC 2 - Area personale}
    \begin{itemize}
        \item Attore primario: utente traduttore/utente admin/utente super admin(zero12).
        \item Precondizione: l'utente è autenticato.
        \item Postcondizione: l'utente accede alla propria area personale.
        \item Scenario principale: l'utente accede alla sua area personale per gestire il proprio profilo.
        \item Sottocasi:
            \begin{itemize}
                \item L'utente vuole uscire dal proprio account [UC 2.1].
                \item L'utente vuole cambiare la propria password [UC 2.2].
            \end{itemize}
    \end{itemize}
    \subsubsection{UC 2.1 - Logout}
        \begin{itemize}
            \item Attore primario: utente traduttore/utente admin/utente super admin(zero12).
            \item Precondizione: l'utente è autenticato.
            \item Postcondizione: l'utente non è più autenticato.
            \item Scenario principale: l'utente vuole uscire dal proprio account.
        \end{itemize}
    \subsubsection{UC 2.2 - Cambio password}
        \begin{itemize}
            \item Attore primario: utente autenticato.
            \item Precondizione: l'utente vuole cambiare la propria password.
            \item Postcondizione: l'utente ha cambiato password.
            \item Scenario principale: l'utente ha neccistà di cambiare la propria password, in genere quando questa non è più sicura.
            \item Flusso di eventi:
                \begin{enumerate}
                    \item L'utente inserisce la vecchia password per riconfermare la sua identità [UC 2.2.1].
                    \item L'utente inserisce la nuova password [UC 2.2.2].
                    \item Per sicurezza il sistema chiede il reinserimento della nuova password [UC 2.2.3].
                \end{enumerate}
        \end{itemize}
        \paragraph{UC 2.2.1 - Inserimento password vecchia}
            \begin{itemize}
                \item Attore primario: utente autenticato.
                \item Precondizione: l'utente ha necessità di cambiare password.
                \item Postcondizione: l'utente ha inserito la vecchia password.
                \item Scenario principale: inserimento vecchia password per modificarla.
                \item Estensioni:
                    \begin{itemize}
                        \item se la password è sbagliata viene visualizzato un messaggio d'errore.
                    \end{itemize}
            \end{itemize}
        \paragraph{UC 2.2.2 - Inserimento password nuova}
            \begin{itemize}
                \item Attore primario: utente autenticato.
                \item Precondizione: l'utente ha necessità di cambiare password ed ha inserito la vecchia password.
                \item Postcondizione: l'utente non autenticato ha inserito la nuova password.
                \item Scenario principale: inserimento nuova password.
                \item Estensioni:
                    \begin{itemize}
                        \item se la password non è conforme agli standard di sicurezza viene visualizzato un messaggio d'errore.
                    \end{itemize}
            \end{itemize}
        \paragraph{UC 2.2.3 - Conferma password}
            \begin{itemize}
                \item Attore primario: utente autenticato.
                \item Precondizione: l'utente ha necessità di cambiare password ed ha inserito la vecchia e la nuova password.
                \item Postcondizione: l'utente non autenticato ha inserito di nuovo la nuova password.
                \item Scenario principale: conferma nuova password per evitare errori.
                \item Estensioni:
                    \begin{itemize}
                        \item se la password non è uguale a quella inserita in precedenza viene visualizzato un messaggio d'errore.
                    \end{itemize}
            \end{itemize}




    



\subsection{UC 4 - Inserimento traduzione}
    \begin{itemize}
    \item Attore primario: utente autenticato qualsiasi.
    \item Precondizione: l'utente vuole inserire una nuova traduzione.
    \item Postcondizione: l'utente ha correttamente inserito una nuova traduzione.
    \item Scenario principale: l'utente clicca sul pulsante "Nuova traduzione", compila i campi dati relativi e clicca il pulsante conferma.
    \item Flusso di eventi:
        \begin{enumerate}
            \item L'utente clicca sul pulsante "Nuova traduzione".
            \item L'utente inserisce la lingua da tradurre [UC 4.1].
            \item L'utente inserisce la lingua in cui bisogna tradurre [UC 4.2].
            \item L'utente inserisce l'espressione da traddure [UC 4.3].
            \item L'utente inserisce l'espressione tradotta [UC 4.4].
        \end{enumerate}
    \end{itemize}
    \subsubsection{UC 4.1 - Inserimento lingua da tradurre}
        \begin{itemize}
            \item Attore primario: utente autenticato qualsiasi.
            \item Precondizione: l'utente ha necessità di inserire una traduzione.
            \item Postcondizione: l'utente ha inserito la lingua da tradurre.
            \item Estensione: l'utente deve inserire obbligatoriamente una lingua da tradurre altrimenti viene visualizzato un messaggio d'errore e non viene salvata la traduzione.
        \end{itemize}
    \subsubsection{UC 4.2 - Inserimento lingua tradotta}
        \begin{itemize}
            \item Attore primario: utente autenticato qualsiasi.
            \item Precondizione: l'utente ha necessità di inserire una traduzione.
            \item Postcondizione: l'utente ha inserito la lingua in cui tradurre.
            \item Estensione: l'utente deve inserire obbligatoriamente una lingua in cui tradurre altrimenti viene visualizzato un messaggio d'errore e non viene salvata la traduzione.
        \end{itemize}
    \subsubsection{UC 4.3 - Inserimento espressione lingua da tradurre}
        \begin{itemize}
            \item Attore primario: utente autenticato qualsiasi.
            \item Precondizione: l'utente ha necessità di inserire una traduzione.
            \item Postcondizione: l'utente ha inserito l'espressione da tradurre.
            \item Estensione: l'utente deve inserire obbligatoriamente almeno un carattere altrimenti viene visualizzato un messaggio d'errore e non viene salvata la traduzione.
        \end{itemize}
    \subsubsection{UC 4.4 - Inserimento espressione lingua tradotta}
        \begin{itemize}
            \item Attore primario: utente autenticato qualsiasi.
            \item Precondizione: l'utente ha necessità di inserire una traduzione.
            \item Postcondizione: l'utente ha inserito l'espressione in cui tradurre.
            \item Estensione: l'utente deve inserire obbligatoriamente almeno un carattere altrimenti viene visualizzato un messaggio d'errore e non viene salvata la traduzione.
        \end{itemize}
\subsection{UC 5 - Revisione traduzione}
    \subsubsection{UC 5.1 - Approvazione}
    \subsubsection{UC 5.2 - Eliminazione}
    Flusso di eventi: utente schiaccia elimina, utente schiaccia conferma elimina
\subsection{UC 6 - Modifica traduzione}
Flusso di eventi: UC6.1 UC6.2 UC6.3 UC6.4   %! Eventualmente togliere il modifica lingue
    \subsubsection{UC 6.1 - Modifica lingua da tradurre}
    \subsubsection{UC 6.2 - Modifica lingua tradotta}
    \subsubsection{UC 6.3 - Modifica espressione lingua da tradurre}
    \subsubsection{UC 6.4 - Modifica espressione lingua tradotta}
\subsection{UC 7 - Elimina traduzione} %! Eliminazione di più traduzioni?
Flusso di eventi: utente schiaccia elimina, utente schiaccia conferma elimina
\subsection{UC 8 - Ricerca}
    \begin{itemize}
        \item Attore primario: utente traduttore, utente admin, utente super admin(zero12).
        \item Precondizione: l'utente all'interno di un progetto ha necessità di ricercare una traduzione.
        \item Postcondizione: l'utente ha come risultato della ricerca tutte le traduzioni che rispettano la query inserita.
        \item Estensione: l'utente deve inserire l'id collegato ad una traduzione
    \end{itemize}
    \subsubsection{UC 8.2 - Ricerca tramite id univoco}
        \begin{itemize}
            \item Attore primario: utente traduttore, utente admin, utente super admin(zero12).
            \item Precondizione: l'utente ha necessità di recercare una traduzione per id univoco della traduzione.
            \item Postcondizione: l'utente ha come risultato della ricerca tutte le traduzioni che rispettano l' id inserito.
            \item Flusso: Inserimento id univoco [8.2.1]
        \end{itemize}
        \paragraph{UC 8.2.1 - Inserimento id univoco}
            \begin{itemize}
                \item Attore primario: utente traduttore, utente admin, utente super admin(zero12).
                \item Precondizione: l'utente ha necessità di inserire un id per ricercare una traduzione.
                \item Postcondizione: l'utente ha come risultato della ricerca tutte le traduzioni che rispettano l'id inserito.
                \item Estensione: l'utente deve inserire l'id univoco
            \end{itemize}
\subsubsection{UC 9 - Filtraggio ricerca}
    \begin{itemize}
        \item Attore primario: utente traduttore, utente admin, utente super admin(zero12).
        \item Precondizione: l'utente ha necessità di filtrare la ricerca effetuata.
        \item Postcondizione: l'utente ha come risultato della ricerca tutte le traduzioni che rispettano i filtri inseriti.
        \item Estensione: l'utente deve selezionare uno tra i seguenti filtri: lingua da tradurre, lingua tradotta, data di creazione, utente traduttore, approvazione, pubblicazione
    \end{itemize}
    \subsubsection{UC 9.1 - Filtraggio per lingua da tradurre}
        \begin{itemize}
            \item Attore primario: utente traduttore, utente admin, utente super admin(zero12).
            \item Precondizione: l'utente ha necessità di filtrare la ricerca effetuata per la lingua da tradurre.
            \item Postcondizione: l'utente ha come risultato della ricerca tutte le traduzioni che rispettano la lingua selezionata precedentemente. 
            \item Estensione: l'utente deve selezionare la lingua da tradurre per eseguire la ricerca filtrata.
        \end{itemize}        
    \subsubsection{UC 9.2 - Filtraggio per lingua tradotta}
        \begin{itemize}
            \item Attore primario: utente traduttore, utente admin, utente super admin(zero12).
            \item Precondizione: l'utente ha necessità di filtrare la ricerca effetuata per la lingua tradotta.
            \item Postcondizione: l'utente ha come risultato della ricerca tutte le traduzioni che rispettano la lingua selezionata precedentemente. 
            \item Estensione: l'utente deve selezionare la lingua tradotta per eseguire la ricerca filtrata.
        \end{itemize}        
    \subsubsection{UC 9.3 - Filtraggio per data di creazione}
        \begin{itemize}
            \item Attore primario: utente traduttore, utente admin, utente super admin(zero12).
            \item Precondizione: l'utente ha necessità di filtrare la ricerca effetuata per la data di creazione.
            \item Postcondizione: l'utente ha come risultato della ricerca tutte le traduzioni che rispettano la data di creazione precedentemente selezionata. 
            \item Estensione: l'utente deve selezionare la data di creazione per eseguire la ricerca filtrata.
        \end{itemize}       
    \subsubsection{UC 9.4 - Filtraggio per utente traduttore}
        \begin{itemize}
            \item Attore primario: utente traduttore, utente admin, utente super admin(zero12).
            \item Precondizione: l'utente ha necessità di filtrare la ricerca effetuata per utente traduttore.
            \item Postcondizione: l'utente ha come risultato della ricerca tutte le traduzioni che rispettano l'utente traduttore precedentemente selezionato. 
            \item Estensione: l'utente deve selezionare l'utente traduttore per eseguire la ricerca filtrata.
        \end{itemize}  
    \subsubsection{UC 9.5 - Filtraggio per approvazione}
        \begin{itemize}
            \item Attore primario: utente traduttore, utente admin, utente super admin(zero12).
            \item Precondizione: l'utente ha necessità di filtrare la ricerca effetuata per approvazione.
            \item Postcondizione: l'utente ha come risultato della ricerca tutte le traduzioni che rispettano l'approvazione precedentemente selezionata. 
            \item Estensione: l'utente deve selezionare l'approvazione per eseguire la ricerca filtrata.
        \end{itemize}  
    \subsubsection{UC 9.6 - Filtraggio per pubblicazione}
        \begin{itemize}
            \item Attore primario: utente traduttore, utente admin, utente super admin(zero12).
            \item Precondizione: l'utente ha necessità di filtrare la ricerca effetuata per pubblicazione.
            \item Postcondizione: l'utente ha come risultato della ricerca tutte le traduzioni che rispettano la pubblicazione precedentemente selezionata. 
            \item Estensione: l'utente deve selezionare la pubblicazione per eseguire la ricerca filtrata.
        \end{itemize} 
\subsection{UC 10 - Pubblicazione di una traduzione}
        \begin{itemize}
            \item Attore primario: utente admin.
            \item Precondizione: l'utente admin ha necessità di pubblicare le traduzioni, le traduzioni possono essere pubblicate solo se passano l'approvazione. 
            \item Postcondizione: l'utente admin ha pubblicato le traduzioni approvate. 
        \end{itemize}
\subsection{UC 11 - Aggiunta cliente}
    \begin{itemize}
        \item Attore primario: utente super admin(zero12).
        \item Precondizione: l'utente ha necessità di aggiungere un nuovo cliente.
        \item Postcondizione: l'utente ha aggiunto il cliente.
        \item Estensione: l'utente deve inserire obbligatoriamente il nome del cliente valido(presente nella lista dei clienti) altrimenti viene visualizzato un messaggio d'errore.
        \item Flusso di eventi:
        \begin{enumerate}
            \item L'utente inserisce il nome del nuovo cliente [UC 11.1].
            \item L'utente inserisce il numero di traduzioni disponibili [UC 11.2].
        \end{enumerate}
    \end{itemize}
    \subsubsection{UC 11.1 - Inserimento nome cliente}
        \begin{itemize}
            \item Attore primario: utente super admin(zero12).
            \item Precondizione: l'utente ha necessità di inserire un cliente, la procedura di inserimento richiede il nome dell'azienda.
            \item Postcondizione: l'utente ha eliminato il cliente.
            \item Estensione: l'utente deve inserire obbligatoriamente il nome del cliente valido(presente nella lista dei clienti) altrimenti viene visualizzato un messaggio d'errore.
        \end{itemize}
    \subsubsection{UC 11.2 - Inserimento numero traduzioni disponibili}
        \begin{itemize}
            \item Attore primario: utente super admin(zero12).
            \item Precondizione: l'utente ha necessità di inserire il numero delle traduzioni disponibili(si chiede una fatturazione per lo spazio su disco richiesto per salvare il numero di traduzioni richieste).
            \item Postcondizione: l'utente ha inserito il numero delle traduzioni.
        \end{itemize}
\subsection{UC 12 - Eliminazione cliente}
    \begin{itemize}
        \item Attore primario: utente super admin(zero12).
        \item Precondizione: l'utente ha necessità di eliminare un cliente.
        \item Postcondizione: l'utente ha eliminato il cliente.
        \item Estensione: l'utente deve inserire obbligatoriamente il nome del cliente valido(presente nella lista dei clienti) altrimenti viene visualizzato un messaggio d'errore.
    \end{itemize}
\subsection{UC 13 - Ricerca traduzione da libreria client}
    \begin{itemize}
        \item Attore primario: libreria client.
        \item Precondizione: libreria client ha necessità di ricercare una traduzione.
        \item Postcondizione: libreria client ha ricevuto la traduzione.
        \item Estensione: libreria client per richiedere la traduzione deve autenticarsi all'API in modo tale da ricevere solo le traduzioni a cui ha accesso.
    \end{itemize}

