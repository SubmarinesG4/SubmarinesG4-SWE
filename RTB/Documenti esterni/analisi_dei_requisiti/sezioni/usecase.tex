\section{Casi d'uso}
\subsection{Scopo}
In questa sezione verranno presentati gli use cases tracciati, che il software dovrà implementare.
\subsection*{UC 1 - Accesso}
    \subsubsection*{UC 1.1 - Prima autenticazione}
    Flusso di eventi: UC1.1.1
        \paragraph*{UC 1.1.1 - Cambio password obbligatorio}
        \paragraph*{UC 1.1.2 - Inserimento indirizzo email obbligatorio}
        \paragraph*{UC 1.1.3 - Inserire dati personali}
    \subsubsection*{UC 1.2 - Autenticazione generica}
        \paragraph*{UC 1.2.1 - Inserimento indirizzo email}
        \paragraph*{UC 1.2.2 - Inserimento password}
    \subsubsection*{UC 1.3 - Password dimenticata}
    Flusso di eventi: UC1.3.1 UC1.3.2 UC1.3.3
        \paragraph*{UC 1.3.1 - Inserimento indirizzo email per recupero}
        \paragraph*{UC 1.3.2 - Inserimento nuova password}
        \paragraph*{UC 1.3.3 - Conferma password}
\subsection*{UC 2 - Logout} %! Mettere area personale come user case?
\subsection*{UC 3 - Cambio password}
    \begin{itemize}
        \item Attore primario: utente autenticato qualsiasi.
        \item Precondizione: l'utente vuole cambiare la propria password.
        \item Postcondizione: l'utente ha cambiato password.
        \item Scenario principale: la password dell'utente in seguito a eventi esterni non è più sicura.
        \item Flusso di eventi:
            \begin{enumerate}
                \item L'utente inserisce la vecchia password per riconfermare la sua identità [UC 3.1].
                \item L'utente inserisce la nuova password [UC 3.2].
                \item Per sicurezza il sistema chiede il reinserimento della nuova password [UC 3.3].
            \end{enumerate}
    \end{itemize}
    \subsubsection*{UC 3.1 - Inserimento password vecchia}
        \begin{itemize}
            \item Attore primario: utente autenticato qualsiasi.
            \item Precondizione: l'utente ha necessità di cambiare password.
            \item Postcondizione: l'utente ha inserito la vecchia password.
            \item Scenario principale: inserimento vecchia password per modificarla.
            \item Estensione: se la password è sbagliata quando viene premuto il tasto "Conferma" viene visualizzato un messaggio d'errore e non viene effettuato il cambio password.
        \end{itemize}
    \subsubsection*{UC 3.2 - Inserimento password nuova}
        \begin{itemize}
            \item Attore primario: utente autenticato qualsiasi.
            \item Precondizione: l'utente ha necessità di cambiare password ed ha inserito la vecchia password.
            \item Postcondizione: l'utente non autenticato ha inserito la nuova password.
            \item Scenario principale: inserimento nuova password.
            \item Estensione: se la nuova password inserita non rispetta gli standard necessari viene visualizzato un messaggio d'errore e non viene effettuato il cambio password.
        \end{itemize}
    \subsubsection*{UC 3.3 - Conferma password}
        \begin{itemize}
            \item Attore primario: utente autenticato qualsiasi.
            \item Precondizione: l'utente ha necessità di cambiare password ed ha inserito la vecchia e la nuova password.
            \item Postcondizione: l'utente non autenticato ha inserito di nuovo la nuova password.
            \item Scenario principale: conferma nuova password per evitare errori.
            \item Estensione: se la password non corrisponde a quella inserita sopra viene visualizzato un messaggio d'errore e non viene effettuato il cambio password.
        \end{itemize}
\subsection*{UC 4 - Inserimento traduzione}
    \begin{itemize}
    \item Attore primario: utente autenticato qualsiasi.
    \item Precondizione: l'utente vuole inserire una nuova traduzione.
    \item Postcondizione: l'utente ha correttamente inserito una nuova traduzione.
    \item Scenario principale: l'utente clicca sul pulsante "Nuova traduzione", compila i campi dati relativi e clicca il pulsante conferma.
    \item Flusso di eventi:
        \begin{enumerate}
            \item L'utente clicca sul pulsante "Nuova traduzione".
            \item L'utente inserisce la lingua da tradurre [UC 4.1].
            \item L'utente inserisce la lingua in cui bisogna tradurre [UC 4.2].
            \item L'utente inserisce l'espressione da traddure [UC 4.3].
            \item L'utente inserisce l'espressione tradotta [UC 4.4].
        \end{enumerate}
    \end{itemize}
    \subsubsection*{UC 4.1 - Inserimento lingua da tradurre}
        \begin{itemize}
            \item Attore primario: utente autenticato qualsiasi.
            \item Precondizione: l'utente ha necessità di inserire una traduzione.
            \item Postcondizione: l'utente ha inserito la lingua da tradurre.
            \item Estensione: l'utente deve inserire obbligatoriamente una lingua da tradurre altrimenti viene visualizzato un messaggio d'errore e non viene salvata la traduzione.
        \end{itemize}
    \subsubsection*{UC 4.2 - Inserimento lingua tradotta}
        \begin{itemize}
            \item Attore primario: utente autenticato qualsiasi.
            \item Precondizione: l'utente ha necessità di inserire una traduzione.
            \item Postcondizione: l'utente ha inserito la lingua in cui tradurre.
            \item Estensione: l'utente deve inserire obbligatoriamente una lingua in cui tradurre altrimenti viene visualizzato un messaggio d'errore e non viene salvata la traduzione.
        \end{itemize}
    \subsubsection*{UC 4.3 - Inserimento espressione lingua da tradurre}
        \begin{itemize}
            \item Attore primario: utente autenticato qualsiasi.
            \item Precondizione: l'utente ha necessità di inserire una traduzione.
            \item Postcondizione: l'utente ha inserito l'espressione da tradurre.
            \item Estensione: l'utente deve inserire obbligatoriamente almeno un carattere altrimenti viene visualizzato un messaggio d'errore e non viene salvata la traduzione.
        \end{itemize}
    \subsubsection*{UC 4.4 - Inserimento espressione lingua tradotta}
        \begin{itemize}
            \item Attore primario: utente autenticato qualsiasi.
            \item Precondizione: l'utente ha necessità di inserire una traduzione.
            \item Postcondizione: l'utente ha inserito l'espressione in cui tradurre.
            \item Estensione: l'utente deve inserire obbligatoriamente almeno un carattere altrimenti viene visualizzato un messaggio d'errore e non viene salvata la traduzione.
        \end{itemize}
\subsection*{UC 5 - Revisione traduzione}
    \subsubsection*{UC 5.1 - Approvazione}
    \subsubsection*{UC 5.2 - Eliminazione}
    Flusso di eventi: utente schiaccia elimina, utente schiaccia conferma elimina
\subsection*{UC 6 - Modifica traduzione}
Flusso di eventi: UC6.1 UC6.2 UC6.3 UC6.4   %! Eventualmente togliere il modifica lingue
    \subsubsection*{UC 6.1 - Modifica lingua da tradurre}
    \subsubsection*{UC 6.2 - Modifica lingua tradotta}
    \subsubsection*{UC 6.3 - Modifica espressione lingua da tradurre}
    \subsubsection*{UC 6.4 - Modifica espressione lingua tradotta}
\subsection*{UC 7 - Elimina traduzione} %! Eliminazione di più traduzioni?
Flusso di eventi: utente schiaccia elimina, utente schiaccia conferma elimina
\subsection*{UC 8 - Ricerca}
    \begin{itemize}
        \item Attore primario: utente traduttore, utente admin, utente super admin(zero12).
        \item Precondizione: l'utente ha necessità di recercare una traduzione.
        \item Postcondizione: l'utente ha come risultato della ricerca tutte le traduzioni che rispettano la query inserita.
        \item Estensione: l'utente deve inserire o una parola o un id collegato ad una traduzione
    \end{itemize}
    \subsubsection*{UC 8.1 - Ricerca tramite parola}
        \begin{itemize}
            \item Attore primario: utente traduttore, utente admin, utente super admin(zero12).
            \item Precondizione: l'utente ha necessità di recercare una traduzione per parola.
            \item Postcondizione: l'utente ha come risultato della ricerca tutte le traduzioni che rispettano la parola inserita.
            \item Flusso: Inserimento parola [8.1.1]
        \end{itemize}
        \paragraph*{UC 8.1.1 - Inserimento parola}
            \begin{itemize}
                \item Attore primario: utente traduttore, utente admin, utente super admin(zero12).
                \item Precondizione: l'utente ha necessità di inserire una parola per recercare una traduzione.
                \item Postcondizione: l'utente ha come risultato della ricerca tutte le traduzioni che rispettano la parola inserita.
                \item Estensione: l'utente deve inserire la parola
            \end{itemize}
    \subsubsection*{UC 8.2 - Ricerca tramite id univoco}
        \begin{itemize}
            \item Attore primario: utente traduttore, utente admin, utente super admin(zero12).
            \item Precondizione: l'utente ha necessità di recercare una traduzione per id univoco della traduzione.
            \item Postcondizione: l'utente ha come risultato della ricerca tutte le traduzioni che rispettano l' id inserito.
            \item Flusso: Inserimento parola [8.2.1]
        \end{itemize}
        \paragraph*{UC 8.2.1 - Inserimento id}
            \begin{itemize}
                \item Attore primario: utente traduttore, utente admin, utente super admin(zero12).
                \item Precondizione: l'utente ha necessità di inserire un id per recercare una traduzione.
                \item Postcondizione: l'utente ha come risultato della ricerca tutte le traduzioni che rispettano l'id inserito.
                \item Estensione: l'utente deve inserire la parola
            \end{itemize}
\subsubsection*{UC 9 - Filtraggio ricerca}
    \begin{itemize}
        \item Attore primario: utente traduttore, utente admin, utente super admin(zero12).
        \item Precondizione: l'utente ha necessità di filtrare la ricerca effetuata.
        \item Postcondizione: l'utente ha come risultato della ricerca tutte le traduzioni che rispettano i filtri inseriti.
        \item Estensione: l'utente deve selezionare uno tra i seguenti filtri: lingua da tradurre, lingua tradotta, data creazione, utente traduttore, approvazione, pubblicazione  ????
    \end{itemize}
    \subsubsection*{UC 9.1 - Filtraggio per lingua da tradurre}
        \begin{itemize}
            \item Attore primario: utente traduttore, utente admin, utente super admin(zero12).
            \item Precondizione: l'utente ha necessità di filtrare la ricerca effetuata per la lingua da tradurre.
            \item Postcondizione: l'utente ha come risultato della ricerca tutte le traduzioni che rispettano la lingua selezionata precedentemente. 
            \item Estensione: l'utente deve selezionare la lingua da tradurre per eseguire la ricerca filtrata.
        \end{itemize}        
    \subsubsection*{UC 9.2 - Filtraggio per lingua tradotta}
        \begin{itemize}
            \item Attore primario: utente traduttore, utente admin, utente super admin(zero12).
            \item Precondizione: l'utente ha necessità di filtrare la ricerca effetuata per la lingua tradotta.
            \item Postcondizione: l'utente ha come risultato della ricerca tutte le traduzioni che rispettano la lingua selezionata precedentemente. 
            \item Estensione: l'utente deve selezionare la lingua tradotta per eseguire la ricerca filtrata.
        \end{itemize}        
    \subsubsection*{UC 9.3 - Filtraggio per data di creazione}
        \begin{itemize}
            \item Attore primario: utente traduttore, utente admin, utente super admin(zero12).
            \item Precondizione: l'utente ha necessità di filtrare la ricerca effetuata per la data di creazione.
            \item Postcondizione: l'utente ha come risultato della ricerca tutte le traduzioni che rispettano la data di creazione precedentemente selezionata. 
            \item Estensione: l'utente deve selezionare la data di creazione per eseguire la ricerca filtrata.
        \end{itemize}       
    \subsubsection*{UC 9.4 - Filtraggio per utente traduttore}
        \begin{itemize}
            \item Attore primario: utente traduttore, utente admin, utente super admin(zero12).
            \item Precondizione: l'utente ha necessità di filtrare la ricerca effetuata per utente traduttore.
            \item Postcondizione: l'utente ha come risultato della ricerca tutte le traduzioni che rispettano l'utente traduttore precedentemente selezionato. 
            \item Estensione: l'utente deve selezionare l'utente traduttore per eseguire la ricerca filtrata.
        \end{itemize}  
    \subsubsection*{UC 9.5 - Filtraggio per approvazione}
        \begin{itemize}
            \item Attore primario: utente traduttore, utente admin, utente super admin(zero12).
            \item Precondizione: l'utente ha necessità di filtrare la ricerca effetuata per approvazione.
            \item Postcondizione: l'utente ha come risultato della ricerca tutte le traduzioni che rispettano l'approvazione precedentemente selezionata. 
            \item Estensione: l'utente deve selezionare l'approvazione per eseguire la ricerca filtrata.
        \end{itemize}  
    \subsubsection*{UC 9.6 - Filtraggio per pubblicazione}
        \begin{itemize}
            \item Attore primario: utente traduttore, utente admin, utente super admin(zero12).
            \item Precondizione: l'utente ha necessità di filtrare la ricerca effetuata per pubblicazione.
            \item Postcondizione: l'utente ha come risultato della ricerca tutte le traduzioni che rispettano la pubblicazione precedentemente selezionata. 
            \item Estensione: l'utente deve selezionare la pubblicazione per eseguire la ricerca filtrata.
        \end{itemize} 
\subsection*{UC 10 - Pubblicazione di una traduzione} %! Pubblicazione di più traduzioni?
\subsection*{UC 11 - Aggiunta cliente}
    \begin{itemize}
        \item Attore primario: utente super admin(zero12).
        \item Precondizione: l'utente ha necessità di aggiungere un nuovo cliente.
        \item Postcondizione: l'utente ha aggiunto il cliente.
        \item Estensione: l'utente deve inserire obbligatoriamente il nome del cliente valido(presente nella lista dei clienti) altrimenti viene visualizzato un messaggio d'errore.
        \item Flusso di eventi:
        \begin{enumerate}
            \item L'utente inserisce il nome del nuovo cliente [UC 11.1].
            \item L'utente inserisce il numero di traduzioni disponibili [UC 11.2].
        \end{enumerate}
    \end{itemize}
    \subsubsection*{UC 11.1 - Inserimento nome cliente}
        \begin{itemize}
            \item Attore primario: utente super admin(zero12).
            \item Precondizione: l'utente ha necessità di inserire un cliente, la procedura di inserimento richiede il nome dell'azienda.
            \item Postcondizione: l'utente ha eliminato il cliente.
            \item Estensione: l'utente deve inserire obbligatoriamente il nome del cliente valido(presente nella lista dei clienti) altrimenti viene visualizzato un messaggio d'errore.
        \end{itemize}
    \subsubsection*{UC 11.2 - Inserimento numero traduzioni disponibili}
        \begin{itemize}
            \item Attore primario: utente super admin(zero12).
            \item Precondizione: l'utente ha necessità di inserire il numero delle traduzioni disponibili(si chiede una fatturazione per lo spazio su disco richiesto per salvare il numero di traduzioni richieste).
            \item Postcondizione: l'utente ha inserito il numero delle traduzioni.
            \item Estensione: ??
        \end{itemize}
\subsection*{UC 12 - Eliminazione cliente}
    \begin{itemize}
        \item Attore primario: utente super admin(zero12).
        \item Precondizione: l'utente ha necessità di eliminare un cliente.
        \item Postcondizione: l'utente ha eliminato il cliente.
        \item Estensione: l'utente deve inserire obbligatoriamente il nome del cliente valido(presente nella lista dei clienti) altrimenti viene visualizzato un messaggio d'errore.
    \end{itemize}

