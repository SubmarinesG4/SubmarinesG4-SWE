\section{Casi d'uso}
\subsection{Scopo}
In questa sezione verranno presentati gli use cases tracciati, che il software dovrà implementare.
\subsection{UC 1 - Accesso}
    \begin{itemize}
        \item Attore primario: utente non autenticato.
        \item Precondizione: l'utente non è autenticato.
        \item Postcondizione: l'utente si registra [UC 1.1] o se è già in possesso delle credenziali effettua l'autenticazione [UC 1.2].
        \item Scenario principale: se l'utente non ha ancora un account dev'essere in possesso di un link di registrazione fornito dall'admin del proprio gruppo [UC 1.1], altrimenti può autenticarsi [UC 1.2].
        \item Scenario alternativo: l'utente già in possesso di un account ha scordato la password e ha bisogno di cambiarla in modo da poter accedere di nuovo [UC 1.3].
    \end{itemize}
    \subsubsection{UC 1.1 - Registrazione}
        \begin{itemize}
            \item Attore primario: utente non autenticato.
            \item Precondizione: l'utente non è registrato.
            \item Postcondizione: l'utente è registrato.
            \item Scenario principale: l'utente ha ricevuto una mail dall'admin del proprio gruppo per potersi e registrare e ha necessità di registrarsi.
            \item Flusso di eventi:
                \begin{enumerate}
                    \item L'utente inserisce il suo indirizzo email [UC 1.1.1].
                    \item L'utente inserisce il suo cognome [UC 1.1.2].
                    \item L'utente inserisce il suo nome [UC 1.1.3].
                    \item L'utente inserisce una password [UC 1.1.4].
                    \item L'utente reinserisce la password [UC 1.1.5].
                \end{enumerate}
        \end{itemize}
        \paragraph{UC 1.1.1 - Inserimento indirizzo email}
            \begin{itemize}
                \item Attore primario: utente non autenticato.
                \item Precondizione: l'utente non è registrato.
                \item Postcondizione: l'utente ha inserito l'indirizzo email.
                \item Scenario principale: l'utente inserisce l'indirizzo email per effettuare la registrazione.
                \item Estensioni:
                    \begin{itemize}
                        \item Se l'indirizzo email è già nel sistema viene visualizzato un messaggio d'errore e la registrazione non viene eseguita.
                    \end{itemize}
            \end{itemize}
        \paragraph{UC 1.1.2 - Inserimento cognome}
            \begin{itemize}
                \item Attore primario: utente non autenticato.
                \item Precondizione: l'utente non è registrato.
                \item Postcondizione: l'utente ha inserito il cognome.
                \item Scenario principale: l'utente inserisce il cognome per effettuare la registrazione.
                \item Estensioni:
                    \begin{itemize}
                        \item Se il cognome risulta una stringa vuota viene visualizzato un messaggio d'errore e la registrazione non viene eseguita.
                    \end{itemize}
            \end{itemize}
        \paragraph{UC 1.1.3 - Inserire nome}
            \begin{itemize}
                \item Attore primario: utente non autenticato.
                \item Precondizione: l'utente non è registrato.
                \item Postcondizione: l'utente ha inserito il nome.
                \item Scenario principale: l'utente inserisce il nome per effettuare la registrazione.
                \item Estensioni:
                    \begin{itemize}
                        \item Se il nome risulta una stringa vuota viene visualizzato un messaggio d'errore e la registrazione non viene eseguita.
                    \end{itemize}
            \end{itemize}
        \paragraph{UC 1.1.4 - Inserimento password}
            \begin{itemize}
                \item Attore primario: utente non autenticato.
                \item Precondizione: l'utente non è registrato.
                \item Postcondizione: l'utente ha inserito una password.
                \item Scenario principale: l'utente inserisce una password per effettuare la registrazione.
                \item Estensioni:
                    \begin{itemize}
                        \item Se la password non risulta conforme agli standard di sicurezza viene visualizzato un messaggio d'errore e la registrazione non viene eseguita.
                    \end{itemize}
            \end{itemize}
        \paragraph{UC 1.1.5 - Reinserimento password}
            \begin{itemize}
                \item Attore primario: utente non autenticato.
                \item Precondizione: l'utente non è registrato.
                \item Postcondizione: l'utente ha reinserito la password.
                \item Scenario principale: l'utente reinserisce la password inserita in [UC 1.1.4] per effettuare la registrazione.
                \item Estensioni:
                    \begin{itemize}
                        \item Se la password non è uguale a quella inserita in [UC 1.1.4] viene visualizzato un messaggio d'errore e la registrazione non viene eseguita.
                    \end{itemize}
            \end{itemize}
    \subsubsection{UC 1.2 - Autenticazione}
        \begin{itemize}
            \item Attore primario: utente non autenticato.
            \item Precondizione: l'utente non è autenticato.
            \item Postcondizione: l'utente è autenticato.
            \item Scenario principale: l'utente inserisce email e password e accede al sistema autenticandosi.
            \item Scenario alternativo: l'utente inserisce una coppia email-password non presente nel sistema.
            \item Flusso di eventi:
                \begin{enumerate}
                    \item L'utente inserisce il suo indirizzo email [UC 1.2.1].
                    \item L'utente inserisce la sua password [UC 1.2.2].
                \end{enumerate}
            \item Estensioni:
                \begin{itemize}
                    \item se la coppia email-password non è presente nel sistema l'autenticazione fallisce e viene visualizzato un messaggio d'errore.
                \end{itemize}
        \end{itemize}
        \paragraph{UC 1.2.1 - Inserimento indirizzo email}
            \begin{itemize}
                \item Attore primario: utente non autenticato.
                \item Precondizione: l'utente non è autenticato.
                \item Postcondizione: l'utente ha inserito l'indirizzo email.
                \item Scenario principale: l'utente inserisce l'indirizzo email per autenticarsi.
            \end{itemize}
        \paragraph{UC 1.2.2 - Inserimento password}
            \begin{itemize}
                \item Attore primario: utente non autenticato.
                \item Precondizione: l'utente non è autenticato.
                \item Postcondizione: l'utente ha inserito la password.
                \item Scenario principale: l'utente inserisce la password per autenticarsi.
            \end{itemize}
    \subsubsection{UC 1.3 - Password dimenticata}
        \begin{itemize}
            \item Attore primario: utente non autenticato.
            \item Precondizione: l'utente non è autenticato.
            \item Postcondizione: l'utente non è autenticato ma ha cambiato la password.
            \item Scenario principale: l'utente ha dimenticato la password e fa richiesta per cambiarla.
            \item Flusso di eventi:
                \begin{enumerate}
                    \item L'utente inserisce l'indirizzo email del proprio account.
                    \item Il sistema invia un email all'indirizzo inserito con il link alla pagina per il cambio della password.
                    \item L'utente inserisce una nuova password.
                    \item L'utente reinserisce la nuova password per confermarla.
                \end{enumerate}
            \item Estensioni:
                \begin{itemize}
                    \item Se l'utente inserisce un'indirizzo email non presente nel sistema non riceverà nessuna email.
                \end{itemize}
        \end{itemize}
        \paragraph{UC 1.3.1 - Inserimento indirizzo email per il reset}
            \begin{itemize}
                \item Attore primario: utente non autenticato.
                \item Precondizione: l'utente non è autenticato.
                \item Postcondizione: l'utente ha inserito l'indirizzo email per il reset.
                \item Scenario principale: l'utente inserisce l'indirizzo email per farsi mandare il link di cambio password.
            \end{itemize}
        \paragraph{UC 1.3.2 - Inserimento nuova password}
            \begin{itemize}
                \item Attore primario: utente non autenticato.
                \item Precondizione: l'utente non è autenticato.
                \item Postcondizione: l'utente ha inserito la password.
                \item Scenario principale: l'utente inserisce la password nella pagina corrispondente al link arrivato via mail.
                \item Estensioni:
                    \begin{itemize}
                        \item Se la password non risulta conforme agli standard di sicurezza viene visualizzato un messaggio d'errore e il reset non viene eseguito.
                    \end{itemize}
            \end{itemize}
        \paragraph{UC 1.3.3 - Reinserimento password}
            \begin{itemize}
                \item Attore primario: utente non autenticato.
                \item Precondizione: l'utente non è autenticato.
                \item Postcondizione: l'utente ha reinserito la password.
                \item Scenario principale: l'utente reinserisce la password nella pagina corrispondente al link arrivato via mail.
                \item Estensioni:
                    \begin{itemize}
                        \item Se la password non è uguale a quella inserita in [UC 1.3.2] viene visualizzato un messaggio d'errore e il reset non viene eseguito.
                    \end{itemize}
            \end{itemize}
\subsection{UC 2 - Area personale}
    \begin{itemize}
        \item Attore primario: utente traduttore/utente admin/utente super admin(zero12).
        \item Precondizione: l'utente è autenticato.
        \item Postcondizione: l'utente accede alla propria area personale.
        \item Scenario principale: l'utente accede alla sua area personale per gestire il proprio profilo.
        \item Sottocasi:
            \begin{itemize}
                \item L'utente vuole uscire dal proprio account [UC 2.1].
                \item L'utente vuole cambiare la propria password [UC 2.2].
            \end{itemize}
    \end{itemize}
    \subsubsection{UC 2.1 - Logout}
        \begin{itemize}
            \item Attore primario: utente traduttore/utente admin/utente super admin(zero12).
            \item Precondizione: l'utente è autenticato.
            \item Postcondizione: l'utente non è più autenticato.
            \item Scenario principale: l'utente vuole uscire dal proprio account.
        \end{itemize}
    \subsubsection{UC 2.2 - Cambio password}
        \begin{itemize}
            \item Attore primario: utente autenticato.
            \item Precondizione: l'utente vuole cambiare la propria password.
            \item Postcondizione: l'utente ha cambiato password.
            \item Scenario principale: l'utente ha neccistà di cambiare la propria password, in genere quando questa non è più sicura.
            \item Flusso di eventi:
                \begin{enumerate}
                    \item L'utente inserisce la vecchia password per riconfermare la sua identità [UC 2.2.1].
                    \item L'utente inserisce la nuova password [UC 2.2.2].
                    \item Per sicurezza il sistema chiede il reinserimento della nuova password [UC 2.2.3].
                \end{enumerate}
        \end{itemize}
        \paragraph{UC 2.2.1 - Inserimento password vecchia}
            \begin{itemize}
                \item Attore primario: utente autenticato.
                \item Precondizione: l'utente ha necessità di cambiare password.
                \item Postcondizione: l'utente ha inserito la vecchia password.
                \item Scenario principale: inserimento vecchia password per modificarla.
                \item Estensioni:
                    \begin{itemize}
                        \item se la password è sbagliata viene visualizzato un messaggio d'errore.
                    \end{itemize}
            \end{itemize}
        \paragraph{UC 2.2.2 - Inserimento password nuova}
            \begin{itemize}
                \item Attore primario: utente autenticato.
                \item Precondizione: l'utente ha necessità di cambiare password ed ha inserito la vecchia password.
                \item Postcondizione: l'utente non autenticato ha inserito la nuova password.
                \item Scenario principale: inserimento nuova password.
                \item Estensioni:
                    \begin{itemize}
                        \item se la password non è conforme agli standard di sicurezza viene visualizzato un messaggio d'errore.
                    \end{itemize}
            \end{itemize}
        \paragraph{UC 2.2.3 - Conferma password}
            \begin{itemize}
                \item Attore primario: utente autenticato.
                \item Precondizione: l'utente ha necessità di cambiare password ed ha inserito la vecchia e la nuova password.
                \item Postcondizione: l'utente non autenticato ha inserito di nuovo la nuova password.
                \item Scenario principale: conferma nuova password per evitare errori.
                \item Estensioni:
                    \begin{itemize}
                        \item se la password non è uguale a quella inserita in precedenza viene visualizzato un messaggio d'errore.
                    \end{itemize}
            \end{itemize}
\subsection{UC 3 - Inserimento traduzione}
    \begin{itemize}
        \item Attore primario: utente traduttore/utente admin/utente super admin(zero12).
        \item Precondizione: l'utente vuole inserire una nuova traduzione.
        \item Postcondizione: l'utente ha inserito una nuova traduzione.
        \item Scenario principale: l'utente clicca sul pulsante "Nuova traduzione", compila i campi dati relativi e clicca il pulsante "Conferma".
        \item Flusso di eventi:
            \begin{enumerate}
                \item L'utente seleziona il progetto.
                \item L'utente clicca sul pulsante "Nuova traduzione".
                \item L'utente seleziona la lingua in cui tradurre [UC 3.1].
                \item L'utente scrive la frase/parola da tradurre [UC 3.2].
                \item L'utente clicca sul pulsante "Conferma".
            \end{enumerate}
        \item Estensioni:
        \begin{itemize}
            \item se l'utente non ha selezionato una lingua in cui tradurre o se non ha inserito la traduzione viene visualizzato un messaggio d'errore.
        \end{itemize}
    \end{itemize}
    \subsubsection{UC 3.1 - Selezione lingua in cui tradurre}
        \begin{itemize}
            \item Attore primario: utente traduttore/utente admin/utente super admin(zero12).
            \item Precondizione: l'utente vuole inserire una traduzione.
            \item Postcondizione: l'utente ha selezionato la lingua in cui tradurre.
            \item Scenario principale: l'utente vuole inserire una traduzione e seleziona la lingua in cui tradurre.
        \end{itemize}
    \subsubsection{UC 3.2 - Inserimento traduzione}
        \begin{itemize}
            \item Attore primario: utente traduttore/utente admin/utente super admin(zero12).
            \item Precondizione: l'utente ha necessità di inserire una traduzione.
            \item Postcondizione: l'utente ha inserito la traduzione.
            \item Scenario principale: l'utente vuole inserire una traduzione e seleziona la lingua in cui tradurre.
        \end{itemize}
\subsection{UC 4 - Revisione traduzione}
    \begin{itemize}
        \item Attore primario: utente traduttore/utente admin/utente super admin(zero12).
        \item Precondizione: l'utente vuole revisionare una traduzione.
        \item Postcondizione: l'utente ha revisionato la traduzione.
        \item Scenario principale: l'utente vuole revisionare una traduzione e la approva [UC 4.1].
        \item Scenario alternativo: l'utente vuole revisionare una traduzione e la mette in "modifica" visto che è sbagliata [UC 4.2].
        \item Flusso di eventi:
            \begin{enumerate}
                \item L'utente seleziona la traduzione che vuole revisionare e clicca sul pulsante "Revisiona".
                \item Sottocasi:
                    \begin{enumerate}
                        \item L'utente decide che la traduzione è approvata [UC 4.1].
                        \item L'utente mette la traduzione in "modifica" [UC 4.2].
                    \end{enumerate}
            \end{enumerate}
    \end{itemize}
    \subsubsection{UC 4.1 - Approvazione}
        \begin{itemize}
            \item Attore primario: utente traduttore/utente admin/utente super admin(zero12).
            \item Precondizione: l'utente ha revisionato una traduzione e la vuole approvare.
            \item Postcondizione: l'utente ha approvato la traduzione.
            \item Scenario principale: l'utente vuole approvare una traduzione.
        \end{itemize}
    \subsubsection{UC 4.2 - Traduzione da modificare}
        \begin{itemize}
            \item Attore primario: utente traduttore/utente admin/utente super admin(zero12).
            \item Precondizione: l'utente vuole mettere in "modifica" una traduzione.
            \item Postcondizione: l'utente ha messo in "modifica" la traduzione.
            \item Scenario principale: l'utente vuole mettere in "modifica" una traduzione visto che è errata.
        \end{itemize}
\subsection{UC 5 - Modifica traduzione}
    \begin{itemize}
        \item Attore primario: utente traduttore/utente admin/utente super admin(zero12).
        \item Precondizione: l'utente vuole modificare una traduzione.
        \item Postcondizione: l'utente ha modificato la traduzione.
        \item Scenario principale: l'utente vuole modificare una traduzione.
        \item Flusso di eventi:
            \begin{enumerate}
                \item L'utente seleziona la traduzione che vuole modificare.
                \item L'utente seleziona tra le lingue in cui è tradotta la frase/parola e sceglie quale vuole modificare [UC 5.1].
                \item L'utente modifica la traduzione [UC 5.2].
            \end{enumerate}
    \end{itemize}
    \subsubsection{UC 5.1 - L'utente seleziona la lingua in cui c'è una traduzione che vuole modificare}
        \begin{itemize}
            \item Attore primario: utente traduttore/utente admin/utente super admin(zero12).
            \item Precondizione: l'utente vuole modificare una traduzione.
            \item Postcondizione: l'utente ha selezionato tra le lingue in cui ci sono traduzioni quale vuole modificare.
            \item Scenario principale: l'utente vuole revisionare una traduzione e seleziona la lingua in cui è tradotta la frase/parola.
            \item Estensioni:
                \begin{itemize}
                    \item se l'utente non seleziona la lingua di cui vuole modificare la traduzione viene visualizzato un messaggio d'errore.
                \end{itemize}
        \end{itemize}
    \subsubsection{UC 5.2 - Inserimento modifica}
        \begin{itemize}
            \item Attore primario: utente traduttore/utente admin/utente super admin(zero12).
            \item Precondizione: l'utente vuole modificare una traduzione.
            \item Postcondizione: l'utente ha inserito la modifica.
            \item Scenario principale: l'utente modifica della traduzione.
        \end{itemize}
\subsection{UC 6 - Elimina traduzione}
    \begin{itemize}
        \item Attore primario: utente traduttore/utente admin/utente super admin(zero12).
        \item Precondizione: l'utente vuole eliminare una traduzione.
        \item Postcondizione: l'utente ha eliminato la traduzione.
        \item Scenario principale: l'utente elimina una traduzione.
        \item Flusso di eventi:
            \begin{enumerate}
                \item L'utente seleziona la traduzione che vuole eliminare.
                \item L'utente clicca sul tasto "Elimina".
                \item Viene visualizzato un messaggio per confermare l'eliminazione.
                \item L'utente clicca sul tasto di conferma eliminazione.
            \end{enumerate}
    \end{itemize}
\subsection{UC 7 - Ricerca}
    \begin{itemize}
        \item Attore primario: utente traduttore/utente admin/utente super admin(zero12).
        \item Precondizione: l'utente all'interno di un progetto ha necessità di ricercare una traduzione.
        \item Postcondizione: l'utente ha come risultato della ricerca tutte le traduzioni che rispettano la query inserita.
        \item Scenario principale: l'utente vuole ricercare una traduzione.
        \item Flusso di eventi:
            \begin{enumerate}
                \item L'utente inserisce l'id [UC 7.1].
                \item Il sistema ritorna tutte le traduzioni con un id corrispondente.
            \end{enumerate}
    \end{itemize}
    \subsubsection{UC 7.1 - Inserimento id univoco}
        \begin{itemize}
            \item Attore primario: utente traduttore/utente admin/utente super admin(zero12).
            \item Precondizione: l'utente ha necessità di inserire un id per ricercare una traduzione.
            \item Postcondizione: l'utente ha come risultato della ricerca tutte le traduzioni che rispettano l'id inserito.
            \item Scenario principale: l'utente inserisce l'id univoco.
        \end{itemize}
\subsection{UC 8 - Filtraggio ricerca}
    \begin{itemize}
        \item Attore primario: utente traduttore/utente admin/utente super admin(zero12).
        \item Precondizione: l'utente ha necessità di filtrare la ricerca effettuata.
        \item Postcondizione: l'utente ha come risultato della ricerca tutte le traduzioni che rispettano i filtri inseriti.
        \item Scenario principale: l'utente vuole filtrare una ricerca.
        \item Sottocasi:
            \begin{itemize}
                \item Filtraggio per data di creazione
                \item Filtraggio per utente traduttore
                \item Filtraggio per approvazione
                \item Filtraggio per in "modifica"
                \item Filtraggio per pubblicazione
            \end{itemize}
    \end{itemize}  
    \subsubsection{UC 8.1 - Filtraggio per data di creazione}
        \begin{itemize}
            \item Attore primario: utente traduttore/utente admin/utente super admin(zero12).
            \item Precondizione: l'utente ha necessità di filtrare la ricerca effettuata per la data di creazione.
            \item Postcondizione: l'utente ha come risultato della ricerca tutte le traduzioni che rispettano la data di creazione precedentemente selezionata. 
            \item Scenario principale: l'utente deve selezionare la data di creazione per eseguire la ricerca filtrata.
        \end{itemize}
    \subsubsection{UC 8.2 - Filtraggio per utente traduttore}
        \begin{itemize}
            \item Attore primario: utente traduttore/utente admin/utente super admin(zero12).
            \item Precondizione: l'utente ha necessità di filtrare la ricerca effettuata per utente traduttore.
            \item Postcondizione: l'utente ha come risultato della ricerca tutte le traduzioni che rispettano l'utente traduttore precedentemente selezionato. 
            \item Scenario principale: l'utente deve selezionare l'utente traduttore per eseguire la ricerca filtrata.
        \end{itemize}
    \subsubsection{UC 8.3 - Filtraggio per approvazione}
        \begin{itemize}
            \item Attore primario: utente traduttore/utente admin/utente super admin(zero12).
            \item Precondizione: l'utente ha necessità di filtrare la ricerca effettuata per approvazione.
            \item Postcondizione: l'utente ha come risultato della ricerca tutte le traduzioni che rispettano l'approvazione precedentemente selezionata. 
            \item Scenario principale: l'utente deve selezionare lo stato della traduzione in "approvata" per eseguire la ricerca filtrata.
        \end{itemize}
    \subsubsection{UC 8.4 - Filtraggio per in "modifica"}
        \begin{itemize}
            \item Attore primario: utente traduttore/utente admin/utente super admin(zero12).
            \item Precondizione: l'utente ha necessità di filtrare la ricerca effettuata per traduzioni in "modifica".
            \item Postcondizione: l'utente ha come risultato della ricerca tutte le traduzioni che sono nella lista da "modificare". 
            \item Scenario principale: l'utente deve selezionare la stato della traduzione in "modifica" per eseguire la ricerca filtrata.
        \end{itemize}
    \subsubsection{UC 8.5 - Filtraggio per pubblicazione}
        \begin{itemize}
            \item Attore primario: utente traduttore/utente admin/utente super admin(zero12).
            \item Precondizione: l'utente ha necessità di filtrare la ricerca effetuata per pubblicazione.
            \item Postcondizione: l'utente ha come risultato della ricerca tutte le traduzioni che rispettano la pubblicazione precedentemente selezionata. 
            \item Scenario principale: l'utente deve selezionare lo stato della traduzione in "pubblicata" per eseguire la ricerca filtrata.
        \end{itemize}
\subsection{UC 9 - Pubblicazione di una traduzione}
        \begin{itemize}
            \item Attore primario: utente admin.
            \item Precondizione: l'utente admin ha necessità di pubblicare le traduzioni, le traduzioni possono essere pubblicate solo se passano l'approvazione. 
            \item Postcondizione: l'utente admin ha pubblicato le traduzioni approvate.
            \item Scenario principale: l'utente pubblica una traduzione.
        \end{itemize}
\subsection{UC 10 - Aggiunta cliente}
    \begin{itemize}
        \item Attore primario: utente super admin(zero12).
        \item Precondizione: l'utente ha necessità di aggiungere un nuovo cliente.
        \item Postcondizione: l'utente ha aggiunto il cliente.
        \item Scenario principale: l'utente inserisce il nuovo cliente.
        \item Flusso di eventi:
        \begin{enumerate}
            \item L'utente inserisce il nome del nuovo cliente [UC 10.1].
            \item L'utente inserisce il numero di traduzioni disponibili [UC 10.2].
        \end{enumerate}
    \end{itemize}
    \subsubsection{UC 10.1 - Inserimento nome cliente}
        \begin{itemize}
            \item Attore primario: utente super admin(zero12).
            \item Precondizione: l'utente ha necessità di inserire un cliente, la procedura di inserimento richiede il nome dell'azienda.
            \item Postcondizione: l'utente ha inserito il nome del cliente.
            \item Scenario principale: l'utente inserisce il nome del cliente.
            \item Estensioni: 
                \begin{itemize}
                    \item l'utente deve inserire obbligatoriamente il nome del cliente valido altrimenti viene visualizzato un messaggio d'errore.
                \end{itemize}
        \end{itemize}
    \subsubsection{UC 10.2 - Inserimento numero traduzioni disponibili}
        \begin{itemize}
            \item Attore primario: utente super admin(zero12).
            \item Precondizione: l'utente ha necessità di inserire il numero delle traduzioni disponibili(si chiede una fatturazione per lo spazio su disco richiesto per salvare il numero di traduzioni richieste).
            \item Postcondizione: l'utente ha inserito il numero delle traduzioni.
            \item Scenario principale: l'utente inscerisce il numero delle traduzioni a disponibilità del cliente.
            \item Estensioni: 
                \begin{itemize}
                    \item l'utente deve inserire obbligatoriamente il numero delle traduzioni disponibili.
                \end{itemize}
        \end{itemize}
\subsection{UC 11 - Eliminazione cliente}
    \begin{itemize}
        \item Attore primario: utente super admin(zero12).
        \item Precondizione: l'utente ha necessità di eliminare un cliente.
        \item Postcondizione: l'utente ha eliminato il cliente.
        \item Scenario principale: l'utente inserisce il nome del cliente.
        \item Estensioni: 
        \begin{itemize}
            \item L'utente deve inserire obbligatoriamente un nome di un cliente valido(presente nella lista dei clienti) altrimenti viene visualizzato un messaggio d'errore.
        \end{itemize}
    \end{itemize}
\subsection{UC 12 - Ricerca traduzione da libreria client}
    \begin{itemize}
        \item Attore primario: libreria client.
        \item Precondizione: libreria client ha necessità di ricercare una traduzione.
        \item Postcondizione: libreria client ha ricevuto la traduzione.
        \item Estensione: libreria client per richiedere la traduzione deve autenticarsi all'API in modo tale da ricevere solo le traduzioni a cui ha accesso.
    \end{itemize}

%! GESTIONE DEI PROGETTI
%! MESSAGGI D'ERRORE

%! CHIEDERE PER RICERCA TRAMITE ID O SE BISOGNA ANCHE FARE LA RICERCA IN TUTTE LE COLONNE DELLA TABELLA
%! ZERO12 PUO' VEDERE MOFICARE CREARE LE TRADUZIONI ECC O SOLO CREARE I TENANT
%! UTENTI SECONDARI 