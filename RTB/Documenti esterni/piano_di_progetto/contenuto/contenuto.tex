%Contenuto del documento
%Introduzione
\section{Introduzione}
\subsection{Scopo del Documento}
Lo scopo di questo documento è fornire un prospetto dettagliato riguardare la pianificazione e le modalità tramite le quali verrà sviluppato il progetto.\\
Il documento tratterà, in ordine:
\begin{itemize}
    \item L'Analisi dei Rischi,
    \item La descrizione del modello di sviluppo adottato,
    \item La suddivisione delle fasi e l'assegnazione dei ruoli
    \item La stima dei costi e delle risorse necessarie allo sviluppo.
\end{itemize}

\subsection{Scopo del Prodotto}
Lo scopo di (NOME-PROGETTO) e di Zero12 è la creazione di una piattaforma in grado di gestire i testi delle localizzazioni di mobile apps e webapps. \\
Il sistema, gestito in modalità multi-tenant, sarà costituito principalmente da un'API\glo{} che permette agli sviluppatori di accedere alle traduzioni dei loro 
testi da inserire all'interno delle apps, e da una webapp di backoffice (CMS) che permette agli amministratori del sistema di accedere al database di traduzioni.

\subsection{Glossario}
\gloDesc

\subsection{Riferimenti}
\subsubsection{Riferimenti Normativi} % DA RIVEDERE
\begin{itemize}
    \item \emph{NormeDiProgetto-v1.0.0};
    \item Regolamento del progetto didattico: \\ \href{https://www.math.unipd.it/~tullio/IS-1/2022/Dispense/PD02.pdf}{\color{blue}https://www.math.unipd.it/~tullio/IS-1/2022/Dispense/PD02.pdf}
    \item Capitolato d'appalto C4: \\ \href{https://www.math.unipd.it/~tullio/IS-1/2022/Progetto/C4.pdf}{\color{blue}https://www.math.unipd.it/~tullio/IS-1/2022/Progetto/C4.pdf}
\end{itemize}

\subsubsection{Riferimenti Informativi} % DA RIVEDERE
\begin{itemize}
    \item \emph{PianoDiQualifica-v1.0.0};
    \item I processi del ciclo di vita del software - Materiale didattico del corso IdS \\ \href{https://www.math.unipd.it/~tullio/IS-1/2022/Dispense/T02.pdf}{\color{blue}https://www.math.unipd.it/~tullio/IS-1/2022/Dispense/T02.pdf}
    \item Gestione di progetto - Materiale didattico del corso IdS \\ \href{https://www.math.unipd.it/~tullio/IS-1/2022/Dispense/T04.pdf}{\color{blue}https://www.math.unipd.it/~tullio/IS-1/2022/Dispense/T04.pdf}
\end{itemize}

\subsection{Scadenze}
Dopo opportune valutazione, il gruppo \emph{Submarines} si impegna a rispettare le seguenti scadenze per lo svolgimento del progetto (NOME-PROGETTO):
\begin{itemize}
    \item \textbf{\RTB{} (RTB)}: settimana dal (DATA) a (DATA);
    \item \textbf{\PB{} (PB)}: settimana dal (DATA) a (DATA);
    \item \textbf{\CA{} (CA)}: settimana dal (DATA) a (DATA);
\end{itemize}

%Analisi dei Rischi
\section{Analisi dei rischi}
L'Analisi dei Rischi è un processo tramite il quale si cerca di prevedere e valutare gli eventuali rischi in cui si può
incorrere durante lo sviluppo di un progetto. La procedura per la gestione di tali rischi può essere suddiviso in 4 attività:
\begin{itemize}
    \item \textbf{Identificazione dei rischi}: Individuazione di eventuali problematiche che possono compromettere l'avanzamento;
    \item \textbf{Analisi dei rischi}: Individuazione delle conseguenze del rischio sul progetto e della probabilità di occorrenza;
    \item \textbf{Piano di contingenza}: Individuazione dei potenziali rischi e delineamento dei passaggi o condotta che il Team deve intraprendere per combatterli.
\end{itemize}


\subsection{Rischi tecnologici} \label{subsection:rischi_tecnologici}
\begin{table}[H]
  \centering
  \renewcommand{\arraystretch}{1.8}
  \rowcolors{2}{gray!100!black!40}{gray!100!black!30}
  \begin{tabular}{p{5.5cm}|p{5cm}|p{5cm}|c}
    \rowcolor[HTML]{1F85DE} 
    \multicolumn{1}{c}{\color[HTML]{FFFFFF}\textbf{Codice}}
    & \multicolumn{1}{c}{\color[HTML]{FFFFFF}\textbf{Descrizione}}
    & \multicolumn{1}{c}{\color[HTML]{FFFFFF}\textbf{Identificazione}}
    & \color[HTML]{FFFFFF}\textbf{Impatto}\\
    \hline
    \textbf{RT1} - Poca esperienza relativa alle tecnologie da utilizzare & Alcuni componenti del gruppo non hanno le conoscenze pressochè avanzate delle tecnologie da utilizzare per un conseguimento del progetto più fluido. & Spetta ai singoli membri del gruppo rilevare le carenze relative alle tecnologie, particolarmente sarà il \roleProjectManagerLow{} colui che dovrà prestarne maggiore interesse. & Alto\\
    \textbf{RT2} - Problemi hardware o software (comprese reti internet) & Possono presentarsi nel caso in cui i dispositivi utilizzati dai singoli membri nell'avanzamento del progetto non funzionano come dovrebbero o sono assenti. & È compito di chi incontra questi problemi di segnalarlo agli altri membri del Team. & Medio  \\
  \end{tabular}
  \caption{Rischi tecnologici}
\end{table}

\subsection{Rischi personali} \label{subsection:rischi_personali}
\begin{table}[H]
    \centering
    \renewcommand{\arraystretch}{1.8}
    \rowcolors{2}{gray!100!black!40}{gray!100!black!30}
    \begin{tabular}{p{5.5cm}|p{5cm}|p{5cm}|c}
        \rowcolor[HTML]{1F85DE}
        \multicolumn{1}{c}{\color[HTML]{FFFFFF}\textbf{Codice}}
        & \multicolumn{1}{c}{\color[HTML]{FFFFFF}\textbf{Descrizione}}
        & \multicolumn{1}{c}{\color[HTML]{FFFFFF}\textbf{Identificazione}}
        & \color[HTML]{FFFFFF}\textbf{Impatto}\\
        \hline
        \textbf{RP1} - Impegni dei singoli membri & Non sempre i componenti del gruppo possono essere disponibili alla partecipazione degli incontri o potrebbero avere delle difficoltà riguardo le tempistiche del lavoro da svolgere & Sarà dovere di ogni membro quello di comunicare settimana per settimana i propri impegni personali, in modo tale da risolvere eventuali situazioni di stallo. & Alto\\
        \textbf{RP2} - Inesperienza lavorativa & Alcuni membri del gruppo non hanno esperienza nel lavorare su progetti di gruppo, più in generale sul rapporto cliente-fornitore. Essendo dunque una nuova esperienza per la maggior parte del Team, queste modalità di lavoro possono portare problemi. & Il \roleProjectManagerLow{} è incaricato di individuare tali problematiche con i singoli membri e di conseguenza capire come poter aiutarli, e massimizzare il contributo di ognuno & Alto  \\
        \textbf{RP3} - Punti di intesa & Si possono presentare situazioni nelle quali diversi membri del gruppo non riescono a trovare un punto d'intesa su un qualsiasi argomento. &  Uno degli interessi primari del gruppo è quello di evitare situazioni del genere, è comunque compito del \roleProjectManagerLow{} quello di gestire il gruppo. & Medio\\
    \end{tabular}
  \caption{Rischi personali}
\end{table}


\subsection{Rischi organizzativi} \label{subsection:rischi_organizzativi}
\begin{table}[H]
    \centering
    \renewcommand{\arraystretch}{1.8}
    \rowcolors{2}{gray!100!black!40}{gray!100!black!30}
    \begin{tabular}{p{5.5cm}|p{5cm}|p{5cm}|c}
        \rowcolor[HTML]{1F85DE}
        \multicolumn{1}{c}{\color[HTML]{FFFFFF}\textbf{Codice}}
        & \multicolumn{1}{c}{\color[HTML]{FFFFFF}\textbf{Descrizione}}
        & \multicolumn{1}{c}{\color[HTML]{FFFFFF}\textbf{Identificazione}}
        & \color[HTML]{FFFFFF}\textbf{Impatto}\\
        \hline
        \textbf{RO1} - Carico del lavoro mal distribuito & Situazione per la quale il lavoro da eseguire è stato suddiviso erroneamente, ad esempio sono state distribuite più attività con un costo dispendioso a diversi membri. Ciò porta dunque a rallentamenti e poca accuratezza su ciò per cui si sta lavorando. & Chiunque ritenga di avere un carico di lavoro molto alto rispetto agli altri componenti deve segnalarlo al più presto. & Medio \\
        \textbf{RO2} - Ritardi con le consegne & Potrebbe capitare che per via di diverse attività relative ai vari corsi universitari sommati agli impegni personali dei singoli, i tempi per il conseguimento del progetto possano allungarsi. & Il gruppo si impegnerà ad evitare questa situazione, rimane comunque compito del \roleProjectManagerLow{} quello di gestire al meglio questo genere di problema. & Basso \\
        \textbf{RO3} - Costi delle attività & Dato che per ogni attività viene stimato un tempo ed un costo, talvolta questi potrebbero risultare inadeguati data l'inesperienza dei membri. & Nel caso in cui un componente finisce prima o dopo il tempo prestabilito o non utilizza la cifra stimata del denaro prefissato deve segnalarlo al \roleProjectManagerLow{}. & Alto\\
    \end{tabular}
  \caption{Rischi organizzativi}
\end{table}

\subsection{Rischi legati ai requisiti} \label{subsection:rischi_legati_ai_requisiti}
\begin{table}[H]
    \centering
    \renewcommand{\arraystretch}{1.8}
    \rowcolors{2}{gray!100!black!40}{gray!100!black!30}
    \begin{tabular}{p{5.5cm}|p{5cm}|p{5cm}|c}
        \rowcolor[HTML]{1F85DE}
        \multicolumn{1}{c}{\color[HTML]{FFFFFF}\textbf{Codice}}
        & \multicolumn{1}{c}{\color[HTML]{FFFFFF}\textbf{Descrizione}}
        & \multicolumn{1}{c}{\color[HTML]{FFFFFF}\textbf{Identificazione}}
        & \color[HTML]{FFFFFF}\textbf{Impatto}\\
        \hline
        \textbf{RR1} - Incomprensione dei requisiti & Il gruppo potrebbe non aver compreso appieno i requisiti del progetto. & Sarà l'azienda proponente a far notare ciò. & Medio \\
        \textbf{RR2} - Presenza del proponente & Potrebbe capitare la situazione in cui il proponente non è presente quanto dovrebbe & Capire subito quanto è disponibile l'azienda proponente & Basso
    \end{tabular}
   \caption{Rischi legati ai requisiti}
\end{table}

%Modello di sviluppo
\section{Modello di sviluppo}
\subsection{Modello incrementale} %Il modello è DA SCEGLIERE
Test

\section{Pianificazione}
